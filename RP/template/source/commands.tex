% ==================================================================================================
% COMANDI DA RIDEFINIRE
% ==================================================================================================

% Nome/Versione/Data del documento
\newcommand{\documentName}{ERRORE}
\newcommand{\documentVersion}{ERRORE}
\newcommand{\documentDate}{ERRORE}

% Editori del documento
\newcommand{\documentEditors}{ERRORE}

% Verificatore del documento
\newcommand{\documentVerifiers}{ERRORE}

% Approvatore del documento
\newcommand{\documentApprovers}{ERRORE}

% Uso del documento
\newcommand{\documentUsage}{ERRORE}

% Destinatari del documento
\newcommand{\documentAddressee}{ERRORE}


% Sommario del documento
\newcommand{\documentSummary}{ERRORE}

% Versione dei documenti
\newcommand{\docVersionGlo}{\textit{v2.0.0}} % Glossario
\newcommand{\docVersionNdP}{\textit{v2.0.0}} % Norme di Progetto
\newcommand{\docVersionSdF}{\textit{v1.0.0}} % Studio di Fattibilità
\newcommand{\docVersionPdP}{\textit{v2.0.0}} % Piano di Progetto
\newcommand{\docVersionPdQ}{\textit{v2.0.0}} % Piano di Qualifica
\newcommand{\docVersionST}{\textit{v1.0.0}} % Specifica Tecnica
\newcommand{\docVersionDdP}{\textit{v1.0.0}} % Definizione di Prodotto
\newcommand{\docVersionAdR}{\textit{v2.0.0}} % Analisi dei Requisiti
\newcommand{\docVersionMU}{\textit{v1.0.0}} % Manuale Utente
\newcommand{\docVersionMS}{\textit{v1.0.0}} % Manuale Sviluppatore

% ==================================================================================================
% COMANDI DA NON RIDEFINIRE
% ==================================================================================================

% Colori per correzione
\definecolor{rosso}{RGB}{220,57,18}
\definecolor{giallo}{RGB}{255,153,0}
\definecolor{blu}{RGB}{102,140,217}
\definecolor{verdeg}{RGB}{100,230,0}
\definecolor{viola}{RGB}{127,245,212}
\definecolor{rame}{RGB}{184,115,51}
\definecolor{verde}{RGB}{0,80,24}
\definecolor{tableRow}{RGB}{230,235,241}
\definecolor{LogoColor}{HTML}{9B0014}

% Colori per forme per grafici a torta e a barre
\definecolor{Grosso}{rgb}{0.75, 0.18, 0.11}
\definecolor{Ggiallo}{rgb}{0.92, 0.78, 0.26}
\definecolor{Garancione}{rgb}{0.95, 0.52, 0.0}
\definecolor{Gazzurro}{rgb}{0.07, 0.58, 0.72}
\definecolor{Gviola}{rgb}{0.58, 0.44, 0.86}
\definecolor{Gverde}{rgb}{0.63, 0.73, 0.42}

% Stile font
\newcommand{\code}[1]{\texttt{\textbackslash{}#1}} % Testo stile codice
\newcommand{\glo}[1]{\textit{#1}\textbf{\tiny G}}
\newcommand{\doc}[1]{\textit{#1}\textbf{\tiny D}}
\newcommand{\es}[1]{(e.g. #1)} % Esempio e.g.

\newcommand{\docNameLP}{\textit{Lettera di Presentazione}} % Manuale Sviluppatore
%Verbali Interni
\newcommand{\VI}{\textit{Verbale Interno del 04-03-2019}}
\newcommand{\VII}{\textit{Verbale Interno del 18-03-2019}}
\newcommand{\VIII}{\textit{Verbale Interno del 08-04-2019}}
%Verbali Esterni
\newcommand{\VE}{\textit{Verbale Esterno del 22-03-2019}}
\newcommand{\VEE}{\textit{Verbale Esterno del 04-04-2019}}


 
% Commenti
\newcommand{\andrea}{Andrea Chinello}
\newcommand{\andreaDel}[1]{\textcolor{rosso}{\st{#1}}}
\newcommand{\singh}{Sukhjinder Singh}
\newcommand{\singhDel}[1]{\textcolor{giallo}{\st{#1}}}
\newcommand{\daniele}{Daniele Scialabba}
\newcommand{\danieleDel}[1]{\textcolor{blu}{\st{#1}}}
\newcommand{\davide}{Davide Tognon}
\newcommand{\davideDel}[1]{\textcolor{verdeg}{\st{#1}}}
\newcommand{\francesco}{Francesco Barbanti}
\newcommand{\francescoDel}[1]{\textcolor{viola}{\st{#1}}}
\newcommand{\valentin}{Valentin Grigoras}
\newcommand{\valentinDel}[1]{\textcolor{rame}{\st{#1}}}
\newcommand{\giacomo}{Giacomo Corrò}
\newcommand{\giacomoDel}[1]{\textcolor{verde}{\st{#1}}}

% Nome del progetto
\newcommand{\projectName}{MegAlexa}
\newcommand{\groupEmail}{\textit{\href{mailto:hexadec.swe@gmail.com}{hexadec.swe@gmail.com}}}
\newcommand{\groupName}{HexaDec}

% Stato del documento
\newcommand{\dStateApprovato}{\textit{Approvato}}
\newcommand{\dStateNonApprovato}{\textit{Non Approvato}}

% Ruoli del progetto
\newcommand{\roleProjectManager}{\textit{responsabile}}
\newcommand{\roleAdministrator}{\textit{amministratore}}
\newcommand{\roleAnalyst}{\textit{analista}}
\newcommand{\roleProgrammer}{\textit{programmatore}}
\newcommand{\roleDesigner}{\textit{progettista}}
\newcommand{\roleVerifier}{\textit{verificatore}}
\newcommand{\roleProjectManagerP}{\textit{responsabile di progetto}}
\newcommand{\roleAdministratorP}{\textit{amministratori}}
\newcommand{\roleAnalystP}{\textit{analisti}}
\newcommand{\roleProgrammerP}{\textit{programmatori}}
\newcommand{\roleDesignerP}{\textit{progettisti}}
\newcommand{\roleVerifierP}{\textit{verificatori}}

% Referenti e committenti (M = master, S = slave)
\newcommand{\proposerName}{zero12}
\newcommand{\commitNameM}{Prof. Tullio Vardanega}
\newcommand{\commitNameS}{Prof. Riccardo Cardin}

% Nome dei documenti
\newcommand{\docNameGlo}{\textit{Glossario}} % Glossario
\newcommand{\docNameNdP}{\textit{Norme di Progetto}} % Norme di Progetto
\newcommand{\docNameSdF}{\textit{Studio di Fattibilità}} % Studio di Fattibilità
\newcommand{\docNameST}{\textit{Specifica Tecnica}} % Specifica Tecnica
\newcommand{\docNameDdP}{\textit{Definizione di Prodotto}} % Definizione di Prodotto
\newcommand{\docNamePdP}{\textit{Piano di Progetto}} % Piano di Progetto
\newcommand{\docNamePdQ}{\textit{Piano di Qualifica}} % Piano di Qualifica
\newcommand{\docNameAdR}{\textit{Analisi dei Requisiti}} % Analisi dei Requisiti
\newcommand{\docNameMU}{\textit{Manuale Utente}} % Manuale Utente
\newcommand{\docNameMS}{\textit{Manuale Sviluppatore}} % Manuale Sviluppatore

% Nome e versione dei documenti
\newcommand{\docNameVersionGlo}{\docNameGlo{} \docVersionGlo\textbf{\tiny D}} % Glossario
\newcommand{\docNameVersionNdP}{\docNameNdP{} \docVersionNdP\textbf{\tiny D}} % Norme di Progetto
\newcommand{\docNameVersionSdF}{\docNameSdF{} \docVersionSdF\textbf{\tiny D}} % Studio di Fattibilità
\newcommand{\docNameVersionPdP}{\docNamePdP{} \docVersionPdP\textbf{\tiny D}} % Piano di Progetto
\newcommand{\docNameVersionPdQ}{\docNamePdQ{} \docVersionPdQ\textbf{\tiny D}} % Piano di Qualifica
\newcommand{\docNameVersionAdR}{\docNameAdR{} \docVersionAdR\textbf{\tiny D}} % Analisi dei Requisiti
\newcommand{\docNameVersionMU}{\docNameMU{} \docVersionMU\textbf{\tiny D}} % Manuale Utente
\newcommand{\docNameVersionMS}{\docNameMS{} \docVersionMS\textbf{\tiny D}} % Manuale Sviluppatore

% Nome delle revisioni
\newcommand{\RR}{\textit{Revisione dei Requisiti}}
\newcommand{\RP}{\textit{Revisione di Progettazione}}
\newcommand{\RQ}{\textit{Revisione di Qualifica}}
\newcommand{\RA}{\textit{Revisione di Accettazione}}


%Nome Capitolati
\newcommand{\CUno}{\textit{Butterfly}}
\newcommand{\CDue}{\textit{Colletta}}
\newcommand{\CTre}{\textit{G\&B}}
\newcommand{\CQuattro}{\textit{MegAlexa}}
\newcommand{\CCinque}{\textit{P2PCS}}
\newcommand{\CSei}{\textit{Soldino}}

%Comando TODO
\newcommand{\todo}{\#TO DO}

%Comandi Capitolati
\newcommand{\BBenvenuto}{blocco testo di benvenuto}
\newcommand{\BTesto}{blocco Testo}
\newcommand{\BFiltro}{blocco Filtro}
\newcommand{\BFeedRSS}{blocco Feed RSS}
\newcommand{\BInstagram}{blocco Instagram}
\newcommand{\BFacebook}{blocco Facebook}
\newcommand{\BLinkedIn}{blocco LinkedIn}
\newcommand{\BMessenger}{blocco Messenger}
\newcommand{\BSveglia}{blocco Sveglia}
\newcommand{\BMail}{blocco Mail}
\newcommand{\BTelegram}{blocco Telegram}
\newcommand{\BSlack}{blocco Slack}
\newcommand{\BCalendario}{blocco Calendraio}
\newcommand{\BYouTube}{blocco YouTube}
\newcommand{\BYouTubeMusic}{blocco YouTube Music}
\newcommand{\BRadio}{blocco Radio}
\newcommand{\BTV}{blocco Programmazione TV}
\newcommand{\BSpotify}{blocco Spotify}
\newcommand{\BCinema}{blocco Cinema}
\newcommand{\BTrasporti}{blocco Trasporti}
\newcommand{\BLista}{blocco Lista}
\newcommand{\BSicurezza}{blocco Sicurezza}
\newcommand{\BMeteo}{blocco Meteo}
\newcommand{\BKindle}{blocco Kindle}


\newcommand{\utente}{\textbf{Ut: }}
\newcommand{\alexa}{\textbf{A: }}
\newcommand{\voice}[1]{\textit{\texttt{#1}}}

%ridefinizione paragrafo. con a capo
\newcommand{\myparagraph}[1]{\paragraph{#1}\mbox{}\\}
\newcommand{\mysubparagraph}[1]{\subparagraph{#1}\mbox{}\\}

%Riga sotto il titolo
\newcommand{\HRule}{\rule{\linewidth}{0.5mm}}

% Descrizione del glossario
\newcommand{\gloDesc}{I termini utilizzati in questo documento potrebbero generare dubbi riguardo al loro significato, richiedendo pertanto una definizione al fine di evitare ambiguità. Tali termini vengono contrassegnati da una G maiuscola finale a pedice della parola. La loro spiegazione è riportata nel \docNameVersionGlo{}.}

% Comandi per la generazione di grafici a torta e a barre
\newcommand{\pie}[3][]{
	\begin{scope}[#1]
		\pgfmathsetmacro{\curA}{90}
		\pgfmathsetmacro{\r}{1}
		\def\c{(0,0)}
		\node[pie title] at (90:1.3) {#2};
		\foreach \v\s in{#3}{
			\pgfmathsetmacro{\deltaA}{\v/100*360}
			\pgfmathsetmacro{\nextA}{\curA + \deltaA}
			\pgfmathsetmacro{\midA}{(\curA+\nextA)/2}
			
			\path[slice,\s] \c
			-- +(\curA:\r)
			arc (\curA:\nextA:\r)
			-- cycle;
			\pgfmathsetmacro{\d}{max((\deltaA * -(.5/50) + 1) , .5)}
			
			\begin{pgfonlayer}{foreground}
				\path \c -- node[pos=\d,pie values,values of \s]{$\v\%$} +(\midA:\r);
			\end{pgfonlayer}
			
			\global\let\curA\nextA
		}
	\end{scope}
}

\newcommand{\legend}[2][]{
	\begin{scope}[#1]
		\path
		\foreach \n/\s in {#2}
		{
			++(0,-5pt) node[\s,legend box] {} +(5pt,0) node[legend label] {\n} % ++(x,y) -> y: spazio tra i vari campi della legenda, x: l'obliquità
		}
		;
	\end{scope}
}