\section{Capitolato C5 – P2PCS}

\subsection{Informazioni generali}
% ------ Inserire qui testo ------
% 
\begin{itemize}
\item \textbf{Nome}: \textit{P2PCS}: Peer 2 Peer Car sharing;\phantom{.}
\item \textbf{Proponente}: \textit{GaiaGO};\phantom{.}
\item \textbf{Committente}: \commitNameM\mbox{} e \commitNameS.
\end{itemize}

\subsection{Descrizione del capitolo}
Il capitolato C5, proposto dall'azienda GaiaGO, richiede di creare un’applicazione di \glo{car sharing} \glo{peer-to-peer}. Lo scopo è mettere a disposizione un sistema per la condivisione di una automobile tra più persone.


\subsection{Finalità di progetto}
L'azienda richiede:
\begin{itemize} 
    \item Progettare e sviluppare un’applicazione mobile per Android che applichi \glo{Gamification} spronando le persone a utilizzare tale servizio;
    \item La possibilità di localizzare le automobili in modo da essere utilizzabili dagli utenti;
    \item L'utilizzo di almeno cinque degli \glo{otto core drive} di \glo{Octalysis}.
\end{itemize}

\subsection{Tecnologie interessate}
\begin{itemize}
    \item \textbf{Octalysis}: Gamification design \glo{framework};
    \item \textbf{Node.js}: piattaforma open source per l'esecuzione di codice JavaScript;
    \item \textbf{\glo{Henshin MOVENS}}: piattaforma open source per gestione della mobilità, delle \glo{città intelligenti} e \glo{IoT}.
\end{itemize}   

\subsection{Conclusioni}
Questo capitolato non ha convinto nessun componente del gruppo in quanto il capitolato presenta diverse lacune informative. Per tale motivo abbiamo ritenuto opportuno non scegliere questo progetto.