\justify \section{Introduzione}

\subsection{Obiettivo del documento}
L’obiettivo dello \docNameVersionSdF\doc{} è fornire le motivazioni che hanno portato alla scelta del \glo{capitolato} C4 MegAlexa rispetto agli altri capitolati proposti.

\subsection{Glossario}
All’interno dei documenti sono presenti termini che possono presentare significati ambigui. Per tale motivo è stato creato il documento  \docNameGlo{} \docVersionGlo\doc{} che conterrà tali termini affiancati dal loro significato. Le parole che sono presenti nel \docNameGlo{} \docVersionGlo{}, presentano una G a pedice.

\subsection{Riferimenti}

\subsubsection{Normativi}

\begin{itemize}
\item \textbf{Norme di Progetto:} \docNameVersionNdP\doc{}\mbox{}.
\end{itemize}

\subsubsection{Informativi}

\begin{itemize}
\item \textbf{Capitolato 1: Butterfly} - monitor per processi CI/CD \\
\url{https://www.math.unipd.it/~tullio/IS-1/2018/Progetto/C1.pdf};\phantom{.}

\item \textbf{Capitolato 2: Colletta} - piattaforma raccolta dati di analisi di testo\\
\url{https://www.math.unipd.it/~tullio/IS-1/2018/Progetto/C2.pdf};\phantom{.}

\item \textbf{Capitolato 3: G\&B} - monitoraggio intelligente di processi DevOps\\
\url{https://www.math.unipd.it/~tullio/IS-1/2018/Progetto/C3.pdf};\phantom{.}

\item \textbf{Capitolato 4: MegAlexa} - arricchitore di skill di Amazon Alexa\\
\url{https://www.math.unipd.it/~tullio/IS-1/2018/Progetto/C4.pdf};\phantom{.}

\item \textbf{Capitolato 5: P2PCS} - piattaforma di peer-to-peer car sharing\\
\url{https://www.math.unipd.it/~tullio/IS-1/2018/Progetto/C5.pdf};\phantom{.}

\item \textbf{Capitolato 6: Soldino} - piattaforma Ethereum per pagamenti IVA\\
\url{https://www.math.unipd.it/~tullio/IS-1/2018/Progetto/C6.pdf}.

\end{itemize}
