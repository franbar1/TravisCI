\section{Capitolato scelto C4 – MegAlexa}

\subsection{Informazioni generali}

\begin{itemize}
\item \textbf{Nome}: \textit{MegAlexa};
\item \textbf{Proponente}: \textit{zero12};
\item \textbf{Committente}: \commitNameM \mbox{} e \commitNameS.
\end{itemize}

\subsection{Descrizione del capitolo}
L'azienda zero12 propone la creazione di una \glo{skill} per l’assistente vocale \glo{Amazon Alexa} in grado di avviare dei \glo{workflow} che vengono creati tramite applicativo web o mobile dall’utente registrato. L’applicativo mobile è  una applicazione \glo{Android} oppure \glo{iOS}.

\subsection{Finalità del progetto}
La principale richiesta è rendere disponibile, all’utente registrato all'applicazione, delle micro-funzioni chiamate \glo{connettori}.
Tali connettori sono inseriti all’interno di un workflow svolto da un controllo vocale. \\
Il progetto utilizzerà l’infrastruttura \glo{Amazon Web Services} in particolare le tecnologie \glo{Lambda}, \glo{API Gateway}, \glo{Aurora Serverless} e \glo{DynamoDB}. La comunicazione dei risultati all’utente avviene tramite l’assistente vocale Amazon Alexa, l’applicazione mobile o web.

\subsection{Tecnologie interessate}
\begin{itemize}
    \item \textbf{\glo{HTML5}, \glo{JavaScript}, \glo{CSS3}, \glo{Bootstrap}}: linguaggi per lo sviluppo di applicazioni web \glo{client-side};
    \item \textbf{\glo{Node.js}}: piattaforma \glo{open source} per l'esecuzione di codice JavaScript \glo{server-side};
    \item \textbf{Amazon Web Services}: insieme di servizi di \glo{cloud computing} utili per il progetto tra cui:
    \begin{itemize}
        \item \textbf{API Gateway}: servizio per creazione, pubblicazione, manutenzione, monitoraggio e protezione delle API su qualsiasi scala;
        \item \textbf{Lambda}: piattaforma di calcolo \glo{serverless} guidata dagli eventi forniti da Amazon;
        \item \textbf{DynamoDB}: \glo{database} non relazionale, orientato a dati valore-chiave e documenti;
        \item \textbf{Aurora Serverless}: servizio di database relazionale;
        \item \textbf{\glo{Amazon Lex}}: servizio per \glo{mappare} i comandi vocali.
    \end{itemize}  
    \item \textbf{Amazon Alexa}: assistente vocale intelligente di Amazon basato su cloud;
    \item \textbf{\glo{Swift}, \glo{Kotlin}}: linguaggi di programmazione mobile, rispettivamente per iOS e Android.
\end{itemize}   

\subsection{Conclusioni}
Il team considera questo capitolato interessante perché, cimentarsi con 
tecnologie recenti e innovative quali quelle fornite da Amazon Web Services,
può arricchire il bagaglio di conoscenze di ogni membro in modo significativo.\\
La possibilità di studiare e utilizzare Amazon Lex è stata incisiva sulla scelta del capitolato.

