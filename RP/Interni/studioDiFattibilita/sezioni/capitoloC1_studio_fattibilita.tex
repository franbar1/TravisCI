\section{Capitolato C1 – Butterfly}

\subsection{Informazioni generali}

\begin{itemize}
\item \textbf{Nome}: \textit{Butterfly};\phantom{.}
\item \textbf{Proponente}: \textit{Imola Informatica};\phantom{.}
\item \textbf{Committente}: \commitNameM\mbox{} e \commitNameS.
\end{itemize}

\subsection{Descrizione del capitolo}
Imola Informatica propone lo sviluppo di una serie di componenti che accentrano e standardizzano segnalazioni da diversi \glo{strumenti di versionamento} e che 
permettono una gestione automatizzata e personalizzabile di esse.

\subsection{Finalità del progetto}

L’azienda propone una soluzione che prevede lo sviluppo di quattro tipologie di componenti che si interfacciano con gli strumenti di versionamento e provvedono a riportare le segnalazioni nella forma desiderata.
\\
Tali tipologie di componenti sono:
\begin{itemize}
    \item \textbf{\glo{Producers}}:  recuperano le segnalazioni e le pubblicano sotto forma di messaggi;
    \item \textbf{\glo{Brokers}}: istanziano e gestiscono i \glo{Topic};
    \item \textbf{\glo{Consumers}}: gestiscono l'abbonamento ai Topic, recuperano i messaggi e li inviano ai destinatari;
    \item \textbf{\glo{Componente custom specifico}}: recupera i messaggi da un Topic e li inoltra alla persona interessata.

\end{itemize}
\item Le componenti devono avere anche altre funzionalità:
\begin{itemize}
    \item Rispettare la metodologia di sviluppo “\glo{The Twelve-Factor App}”;
    \item Esporre delle \glo{API REST};
    \item Essere supportate da \glo{test unitari} e da \glo{test di integrazione};
    \item Essere istanziate tramite la tecnologia \glo{Docker}.
    
\end{itemize}

\subsection{Tecnologie interessate}
\begin{itemize}
    \item \textbf{\glo{Java}, \glo{Python}, Node.js}: utilizzate per lo sviluppo dei componenti;
    \item \textbf{\glo{Apache Kafka}}: piattaforma open source di \glo{stream processing} utilizzabile come Broker;
    \item \textbf{Docker}: tecnologia di containerizzazione utilizzata per istanziare i componenti;
    \item \textbf{\glo{REST}}: stile architetturale software per \glo{sistemi distribuiti}.
\end{itemize}   

\subsection{Conclusioni}
Il gruppo ha dimostrato grande interesse nella proposta del capitolato in quanto permette di lavorare con tecnologie molto diffuse nell’ambito lavorativo.
\\Il fattore decisivo che non ha reso il primo capitolato come prima scelta del gruppo è stata la ripetitività nello sviluppo di componenti simili tra loro.

