\section{Processi Organizzativi}

\subsection{Gestione di processo}
In questa sezione vengono definite le norme che regolano le comunicazioni tra membri del gruppo con parti esterne.
\subsubsection{Comunicazioni interne}
Per le comunicazioni interne viene adottato un \glo{workspace} di \glo{Slack}, 
strumento molto versatile grazie anche alle numerose integrazioni con altre piattaforme, tra cui Git. \\
In questo workspace, sono stati creati vari canali per gestire al meglio le comunicazioni:
\begin{itemize}
    \item \textbf{commit}: contiene tutte le notifiche relative ai commit che il gruppo fa sulla repository;
    \item \textbf{general}: contiene le discussioni di carattere generale sull'organizzazione del progetto, la scelta degli strumenti o per perdere decisioni in modo rapido;
    \item \textbf{incontri}: contiene le discussioni relative ai posti e orari di incontro del gruppo;
    \item \textbf{random}: contiene discussioni non strettamente inerenti al progetto;
    \item \textbf{materiali}: contiene i documenti prodotti e le guide per l'utilizzo degli strumenti necessari allo svolgimento del progetto, per consentire una più facile discussione;
    \item \textbf{divisionelavoro}: contiene le discussioni relative alla divisione del lavoro tra i membri del gruppo.
\end{itemize}
\subsubsection{Comunicazioni esterne}
In questa sezione vengono esposte le norme che regolano le comunicazioni con soggetti esterni al gruppo, nello specifico:
\begin{itemize}
    \item la proponente \proposerName{}, con referente Stefano Dindo.
    \item \commitNameM{}, \commitNameS{}, ai quali verrà fornita tutta la documentazione richiesta in ciascuna revisione di avanzamento.
\end{itemize}
Le comunicazioni esterne avvengono unicamente per mezzo scritto. Queste devono avvenire esclusivamente attraverso l'indirizzo mail del gruppo: 
\begin{center}
    \groupEmail
\end{center}
Ogni membro del gruppo ha accesso all'account di posta elettronica.

\subsection{Gestione delle riunioni}
Le riunioni possono essere interne o esterne. All'inizio di ogni riunione il \roleProjectManager{} nomina a turno un segretario che si occuperà di prendere nota su ciò che viene discusso e successivamente redigere il verbale. 
Inoltre ha l'onere di far rispettare l'ordine del giorno.

\subsubsection{Riunioni interne}
La partecipazione alle riunioni interne, che avverranno principalmente di persona, è permessa solamente ai membri del gruppo \groupName{}. 
Al \roleProjectManager{} spetta il compito di stilare preventivamente l'ordine del giorno da discutere, fissare una data in accordo con tutti i membri del gruppo e approvare il verbale redatto dal segretario. \\
Affinché una riunione sia valida, è richiesta la partecipazione di almeno cinque membri del gruppo, i quali sono tenuti a presentarsi in orario, segnalare eventuali ritardi o assenze.

\subsubsection{Riunioni esterne}
Le riunioni esterne coinvolgono i membri del gruppo \groupName{} e uno o più soggetti esterni, appartenenti all'azienda proponente. \\
Queste si possono svolgere:
\begin{itemize}
    \item nella sede della proponente; 
    \item attraverso Google Hangouts.
\end{itemize}
In accordo con il rappresentante dell'azienda proponente, eventuali comunicazioni avverranno via e-mail. \\
Per quanto riguarda gli incontri, il gruppo si impegna a comunicare con anticipo al \roleProjectManager{} Stefano Dindo la volontà di effettuare una conference call.
\subsubsection{Verbali delle riunioni}
Al termine di ogni riunione il segretario dovrà redigere il relativo verbale, rispettando il seguente schema:
\begin{enumerate}
    \item \textbf{Informazioni generali}:
        \begin{itemize}
            \item luogo e data dell'incontro;
            \item orario di inizio e di fine incontro;
            \item nominativi dei partecipanti interni;
            \item nominativi dei partecipanti esterni;
            \item nominativo del segretario.
        \end{itemize}
    \item \textbf{Ordine del giorno}: deciso dal \roleProjectManager{}. Si andrà a definire l'ordine del giorno di ogni incontro, in modo tale che possa essere consultato da ogni membro del gruppo;
    \item \textbf{Resoconto}: riassunto redatto dal segretario secondo i punti dell'ordine del giorno, precisando le decisioni prese. Potranno essere presenti anche tematiche non inerenti all'ordine del giorno, ma che sono state trattate durante lo svolgimento della riunione.
\end{enumerate}
\paragraph{Nomenclatura}
Al fine di identificare in modo univoco un verbale, la nomenclatura da adottare per ogni verbale è la seguente: 
                    \begin{center}\textbf{VER-TIPO-DATA}\end{center}
dove 
\begin{itemize}
    \item \textbf{VER} sta per verbale;
    \item \textbf{TIPO} indica la tipologia del verbale, che può essere:
        \begin{itemize}
            \item \textbf{I}: verbale della riunione interna;
            \item \textbf{E}: verbale della riunione esterna.
        \end{itemize}
    \item \textbf{DATA} è la data del verbale.
    \end{itemize}

\subsection{Processi di pianificazione}
In questa sezione sono presentati tutti i sistemi utilizzati dal gruppo per pianificare e coordinare lo svolgimento del progetto. \\
Ogni membro è tenuto a consultare e utilizzare attivamente questi strumenti, in modo da tener traccia del lavoro svolto.

\subsubsection{Ruoli di progetto}
I differenti ruoli verranno assegnati ai membri del gruppo a rotazione ad ogni raggiungimento di una scadenza terminale, garantendo la continuità delle attività in corso. Ogni membro dovrà ricoprire tutti i ruoli, per un periodo significativo, durante lo svolgimento del progetto. I ruoli sono i seguenti:
\begin{itemize}
    \item \textbf{Responsabile}: gestisce le risorse umane, i rischi, la pianificazione, il controllo, il coordinamento e le relazioni esterne.\\ 
            Le sue responsabilità sono l'approvazione dell'emissione di documenti, l'elaborazione e l'emanazione di piani e scadenze, il coordinamento del gruppo e delle attività.\\ 
            Ha il compito di rappresentante del progetto presso l'azienda proponente;
    \item \textbf{Amministratore}: si occupa dell'efficienza e dell'operatività dell'ambiente di sviluppo.\\
            A lui viene affidato il compito di gestione del controllo della configurazione del prodotto, del versionamento e della documentazione di progetto;
    \item \textbf{Analista}: si occupa dell'attività di analisi, della redazione dello \docNameSdF{} e dell'\docNameAdR{}.\\ 
            Ha il compito di capire a fondo il problema definendo i requisiti espliciti ed impliciti;
    \item \textbf{Progettista}: effettua lo studio di fattibilità del prodotto, costruisce l'architettura in termini di efficienza ed efficacia a partire dal lavoro dell'\roleAnalyst. Redige, inoltre, la specifica tecnica del progetto.
    \item \textbf{Programmatore}: gestisce l'attività di codifica del prodotto e le componenti di ausilio necessarie all'esecuzione delle prove di verifica e validazione; deve attenersi alle specifiche fornite dal \roleProjectManager{}.\\ 
            Punto focale del suo ruolo è produrre un codice manutenibile nel tempo;
    \item \textbf{Verificatore}: si occupa delle attività di verifica e validazione, partecipando all'intero ciclo di vita. \\
            Redige la parte retrospettiva del \docNamePdQ{}, ossia illustra l'esito e la completezza delle verifiche e delle prove effettuate.
\end{itemize}

\subsubsection{Pianificazione}
Il gruppo ha scelto di appoggiarsi al servizio di gestione delle \glo{issues} fornito da GitLab, il quale permette di:
\begin{itemize}
    \item inserire issue;
    \item attribuire un titolo e una descrizione;
    \item aggregare issues in una milestone;
    \item assegnare una issue da risolvere ad uno o più membri del gruppo;
    \item commentare una issue;
    \item assegnare delle etichette alle issues.
\end{itemize}
Il \roleProjectManager{} ha il compito di creare le issues e assegnarle ad uno o più membri del gruppo. 
Ogni issue deve possedere delle etichette, che identificano la priorità e il tipo di task da svolgere e una data ultima per il completamento.\\
Il gruppo ha scelto di utilizzare unicamente GitLab in modo da avere una situazione più chiara possibile in un unico spazio di lavoro.

\subsubsection{Coordinameto}
Il gruppo adotta il canale Slack "divisionelavoro" per coordinare al meglio la divisione del lavoro tra membri del gruppo. 
Il \roleProjectManagerP{} si occuperà di dividere il carico di lavoro in modo equo tra i vari componenti del team e comunicherà tali scelte nel canale Slack apposito. Ogni membro dovrà leggere attentamente le istruzioni dettate dal \roleProjectManager{} e, in caso di problemi, riferirli.

\subsection{Versionamento}
Per il versionamento il gruppo ha scelto di utilizzare il software Git, attraverso la piattaforma GitLab.
Il gruppo ha maturato questa scelta in quanto il sistema era già conosciuto da quasi tutti i membri. 
Inoltre, il sistema di versionamento distribuito Git riduce il rischio di perdita di dati. Ogni membro del gruppo è tenuto a tenere aggiornata la repository con le modifiche fatte; per facilitare le procedure e mantenere sempre aggiornato sia l'ambiente di lavoro locale che la repository. \\
Viene inoltre adottato l'editor Visual Studio Code che fornisce un'integrazione con il sistema Git.

\subsection{Documentazione}
\subsubsection{ \LaTeX{} }
Per la stesura dei documenti il gruppo ha scelto l'utilizzo di \LaTeX{} in quanto garantisce una migliore qualità tipografica dei documenti rispetto a un normale word processor.\\ 
Il gruppo ha adottato come editor Visual Studio Code, che offre l'estensione \LaTeX{} Workshop, preferito ad altri editor nativi per il linguaggio \LaTeX{} in quanto permette di tener traccia delle modifiche fatte da ogni membro nei file della repository Git.

\subsubsection{Script}
Per facilitare e automatizzare il lavoro di verifica e creazione dei documenti, sono stati adottati i seguenti script:
\begin{itemize}
    \item \textbf{gulpease.php}
    
    Calcola l'\glo{indice di Gulpease} per ogni sezione di un documento; questo script viene lanciato durante la verifica di un documento da parte del \roleVerifier{}, in modo da garantire almeno la accettabilità di tale documento;
    \item \textbf{glossarize.php}
    
    Questo script crea il glossario, prendendo in input un file che abbia una struttura "termine , descrizione" e produce le sezioni del documento \docNameGlo{}.
\end{itemize}
\subsection{Formazione}
\subsubsection{Formazione dei membri del gruppo}
Ogni membro del gruppo è tenuto a formarsi autonomamente per padroneggiare al meglio le tecnologie che verranno utilizzate nel corso del progetto.\\
Al fine di facilitare la formazione, i membri del gruppo sono tenuti a condividere le conoscenze già possedute o a realizzare, in piena libertà, delle linee guida informali.
\subsubsection{Guide utilizzate}
La seguente documentazione dovrà essere consultata e appresa: 
\begin{itemize}
    \item per l'utilizzo di Slack: \url{https://get.slack.help/hc/en-us/categories/200111606-Using-Slack};
    \item per l'utilizzo di LaTeX: \url{https://www.latex-project.org};
    \item per l'utilizzo del software Git: \url{https://git-scm.com/docs};
    \item per l'utilizzo del GitLab: \url{https://docs.gitlab.com}.
\end{itemize}