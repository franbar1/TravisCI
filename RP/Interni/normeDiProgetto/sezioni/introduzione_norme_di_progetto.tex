\justify \section{Introduzione}

\subsection{Scopo del documento}
Lo scopo del documento è fornire una linea guida riguardo il metodo di lavoro che il gruppo deve seguire per l'intera durata del progetto.
Il documento contiene un insieme di regole per la gestione di processi, attività e task fatti su misura per il
progetto. In questo modo ogni componente del gruppo dovrà seguire un metodo di lavoro comune
facilitando il coordinamento ed evitando disomogeneità all'interno del
progetto.\\ 
Questo documento sarà soggetto a modifiche.

\subsection{Scopo del prodotto}
La principale richiesta del progetto è rendere disponibile, all'utente registrato all'applicazione, delle micro-funzioni chiamate \glo{connettori}.
Tali connettori sono inseriti all'interno di un \glo{workflow} svolto da un controllo vocale. \\
Il progetto utilizzerà l'infrastruttura \glo{Amazon Web Services}, in particolare le tecnologie \glo{Lambda}, \glo{API Gateway}, \glo{Aurora Serverless} e \glo{DynamoDB}. La comunicazione dei risultati all'utente avviene tramite l'assistente vocale \textit{Amazon Alexa\glo}, l'applicazione mobile o web.
L'applicativo dovrà essere multilingua rendendo disponibile la lingua italiana e quella inglese.
Inoltre la creazione di routine dovrà essere univoca per ogni utente, ovvero se l'utente A crea una routine che si chiama "Buongiorno", anche
l'utente B può creare una routine che si chiama "Buongiorno", ma le due routine potrebbero avere workflow diversi.

\subsection{Glossario}
All'interno dei documenti sono presenti termini che possono presentare significati ambigui. 
Per tale motivo è stato creato il documento  \docNameVersionGlo{} che conterrà tali termini 
affiancati dal loro significato. Le parole che sono presenti nel \docNameVersionGlo{} 
presentano una "G" a pedice.  

\subsection{Riferimenti}
\subsubsection{Normativi}
\begin{itemize}
    \item \textbf{Capitolato C4 - \projectName{}}\\ \url{https://www.math.unipd.it/~tullio/IS-1/2018/Progetto/C4.pdf}
    \item \textbf{VER-E-22-03-2019} ovvero il verbale tenuto il 22-03-2019 tramite conference call con il programma Google Hangouts. 
\end{itemize}
\subsubsection{Informativi}
\begin{itemize}
    \item \textbf{Standard ISO 12207}\\ \url{ https://www.math.unipd.it/~tullio/IS-1/2009/Approfondimenti/ISO\_12207-1995.pdf}
    \item \textbf{Documentazione Amazon API Gateway} \\\url{https://aws.amazon.com/it/api-gateway/}
    \item \textbf{Documentazione AWS Lambda} \\\url{https://aws.amazon.com/it/lambda/}
    \item \textbf{Documentazione Amazon Aurora Serverless} \\\url{https://aws.amazon.com/it/rds/aurora/serverless/}   
\end{itemize}