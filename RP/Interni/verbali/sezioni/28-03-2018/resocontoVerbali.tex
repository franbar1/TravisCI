\section{Resoconto}
\subsection{Scelta del nome e del logo}
Ogni membro del gruppo ha proposto uno o più nomi: sulla base di una votazione è stato scelto HexaDec e creato il logo. Sulla base di questo nome`e stata creata anche la casella di posta elettronica \groupEmail (non presente nell’ordine del giorno).

\subsection{Analisi dei capitolati}
Al fine di conoscere in dettaglio i capitolati proposti, tre membri del gruppo hanno analizzato alcuni aspetti, come prodotto finale da sviluppare, tecnologie utilizzate, interesse nell’apprenderle. In particolare:
\begin{itemize}
    \item Daniele ha analizzato il capitolato C3 e C4;
    \item Singh ha analizzato il capitolato C1 e C2;
    \item Francesco ha analizzato il capitolato C5 e C6.
\end{itemize}
A turno ognuno di loro ha presentato al gruppo la proprio analisi e si è scelto il capitolato C4, perché ritenuto interessante dal punto di vista tecnologico.

\subsubsection{Analisi dei documenti da produrre per RR}
Al fine di aver un’idea generale dei documenti da produrre per la Revisione dei Requisiti,Davide e Giacomo hanno presentato al gruppo le loro ricerche.

\subsubsection{AScelta degli strumenti di sviluppo e comunicazione}
Il gruppo ha scelto gli strumenti da utilizzare per lo sviluppo del progetto e per la comunicazione e coordinazione. Le scelte sono state fatte sulla base di esperienze personali o feedback da altri gruppi.