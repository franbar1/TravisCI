\section{Informazioni generali}
\begin{itemize}
    \item Luogo e data: Aula 1BC45, Torre Archimede, Padova, 11 Marzo 2019.
    \item Orario: 13.40 - 15.30
    \item Partecipanti interni: Valentin Grigoras, Francesco Barbanti, Daniele Scialabba, Sukhjinder Singh, Davide Tognon, Andrea Chinello, Giacomo Corrò.
    \item Partecipanti esterni: nessuno.
    \item Segretario: Davide Tognon.
\end{itemize}

\section{Ordine del giorno}
\begin{itemize}
    \item Scelta del nome del gruppo e del logo
    \item Analisi dei capitolati
    \item Analisi dei documenti da produrre per RR
    \item Scelta degli strumenti di sviluppo e comunicazione
\end{itemize}

\section{Resoconto}
\subsection{Scelta del nome e del logo}
Ogni membro del gruppo ha proposto uno o pi`u nomi: sulla bse di una votazione`e statoscelto HexaDec e creato il logo. Sulla base di questo nome`e stata creata anche la caselladi posta elettronica \groupEmail (non presente nell’ordine del giorno).
\subsection{Analisi dei capitolati}
Al fine di conoscere in dettaglio i capitolati proposti, tre membri del gruppo hanno ana-lizzato alcuni aspetti, come prodotto finale da sviluppare, tecnologie utilizzate, interessenell’apprenderle. In particolare:
\begin{itemize}
    \item Daniele ha analizzato il caitolato C3 e C4;
    \item Singh ha analizzato il caitolato C1 e C2;
    \item Francesco ha analizzato il caitolato C5 e C6.
\end{itemize}
A turno ognuno di loro ha presentato al gruppo la proprio analisi e si è scelto il caitolato C4, perché ritenuto interessante dal punto di vista tecnogico.

\subsubsection{Analisi dei documenti da produrre per RR}
Al fine di aver un’idea generale dei documenti da produrre per la Revisione dei Requisiti,Davide e Giacomo hanno presentato al gruppo le loro ricerche.
\subsubsection{AScelta degli strumenti di sviluppo e comunicazione}
Il gruppo ha scelto gli strumenti da utilizzare per lo sviluppo del progetto e per la comu-nicazione e coordinazione. Le scelte sono state fatte sulla base di esperienze personali ofeedback da altri gruppi.