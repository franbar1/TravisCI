\section{\quad$C\quad$}
\subsection{Capitolato}
\index{Capitolato}
Il capitolato è un documento tecnico, in genere allegato ad un contratto di appalto, che vi fa riferimento per definire in quella sede le specifiche tecniche delle opere che andranno ad eseguirsi per effetto del contratto stesso.

\subsection{Car sharing}
\index{Car sharing}
Il car-sharing è un servizio di autonoleggio essenzialmente a breve termine, dove le autovetture sono messe a disposizione da un’azienda e distribuite nei centri urbani.

\subsection{Caso d’uso}
\index{Caso d’uso}
Un caso d’uso è un insieme di scenari (sequenze di azioni) che hanno in comune uno scopo finale (obiettivo) per un utente (attore).

\subsection{Ciclo di vita}
\index{Ciclo di vita}
Il ciclo di vita del software, in informatica, e in particolare nell'ingegneria del software, si riferisce al modo in cui una metodologia di sviluppo scompone l'attività di realizzazione di prodotti software in sottoattività fra loro coordinate, il cui risultato finale è il prodotto stesso e tutta la documentazione ad esso associata: fasi tipiche includono lo studio o analisi, la progettazione, la realizzazione, il collaudo, la messa a punto, la manutenzione e l'estensione, il tutto ad opera di uno o più sviluppatori software.


\subsection{Città intelligenti}
\index{Città intelligenti}
Una città può essere definita smart city quando gli investimenti in capitale umano e sociale e nelle infrastrutture tradizionali (mobilità e trasporti) e moderne alimentano uno sviluppo economico sostenibile ed una elevata qualità della vita con una gestione saggia delle risorse naturali, attraverso un metodo di governo partecipativo.

\subsection{Client-side}
\index{Client-side}
Il termine Client-Side indica le operazioni di elaborazione effettuate da un client in un'architettura client-server.

\subsection{Cloud computing}
\index{Cloud computing}
Cloud Computing è un termine usato per descrivere una varietà di concetti informatici che coinvolge un largo numero di computer collegati tramite una connessione in tempo reale come Internet.

\subsection{Componente custom specifico}
\index{Componente custom specifico}
Indica un componente, una funzione o un dispositivo progettato e realizzato su misura in base alle necessità dell'acquirente o della funzione che è destinato ad assolvere.

\subsection{Connettori}
\index{Connettori}
In architettura del software, un connettore viene utilizzato per descrivere un modulo o un meccanismo software la cui funzione è quella di consentire la comunicazione e la cooperazione fra altri moduli, spesso incapsulando le dipendenze strutturali e funzionali fra di essi.

\subsection{CSS3}
\index{CSS3}
Versione 3 di CSS, che aumenta il supporto per animazioni ed elementi interattivi.

\subsection{Cubit}
\index{Cubit}
Cubits era una piattaforma Bitcoin multiuso per acquistare, scambiare, archiviare e accettare Bitcoin. Cubits offriva servizi di elaborazione dei pagamenti in Bitcoin europei con una piattaforma sicura e intuitiva per acquistare, scambiare, archiviare e accettare Bitcoin. 

