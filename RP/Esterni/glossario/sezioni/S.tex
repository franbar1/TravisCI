\section{\quad$S\quad$}

\subsection{Script}
\index{Script}
Un tipo particolare di programma, scritto in una particolare classe di linguaggi di programmazione, detti linguaggi di scripting.

\subsection{Scripting}
\index{Scripting}
Si indica con Scripting un linguaggio di programmazione interpretato,  destinato in genere a compiti di automazione del sistema operativo (batch) o delle applicazioni (macro), o a essere usato all’interno delle pagine web.  Scripting deriva dalla  parola  inglese  script  che  significa  “copione”.   La  parola  scripting  significa  quindi  “scrivere  il  copione”, dove copione identifica le istruzioni che il processore deve eseguire.

\subsection{SCSS}
\index{SCSS}
Acronimo per Sassy Cascading Style Sheets; indica un soprainsieme della sintassi CSS3.

\subsection{Server-side}
\index{Server-side}
Lato server: fa riferimento a operazioni compiute dal server in un ambito client-server.

\subsection{Serverless}
\index{Serverless}
Con Serverless si intenendono le architetture che consentono di creare ed eseguire applicazioni e servizi senza doverne gestire l’infrastruttura.

\subsection{Sistema embedded}
\index{Sistema embedded}
Un sistema embedded è un sistema di elaborazione specializzato integrato in un dispositivo fisico in modo da controllare le funzioni tramite un apposito programma software dedicato.


\subsection{Sistemi distribuiti}
\index{Sistemi distribuiti}
Un sistema distribuito è costituito da un insieme di entità autonome(componenti software e hardware) spazialmente separate che comunicano e coordinano tra loro le loro azioni attraverso scambio di messaggi.

\subsection{Skill}
\index{Skill}
Una Skill è una capacità o abilità di Alexa. Alexa mette a disposizione delle skill di default (come ad esempio la possibilità di riprodurre musica), inoltre i programmatori possono aggiungerne delle altre configurandone di nuove.

\subsection{Slack}
\index{Slack}
Slack è una piattaforma di messaggistica per team che integra insieme diversi canali di comunicazione in un unico servizio.


\subsection{Smart contracts}
\index{Smart contracts}
Gli Smart Contacts sono protocolli informatici che facilitano, verificano, o fanno rispettare, la negoziazione o l'esecuzione di un contratto, permettendo talvolta la parziale o la totale esclusione di una clausola contrattuale.

\subsection{SPICE}
\index{SPICE}
Software Process Improvement and Capability Determination (SPICE), è un insieme di documenti tecnici standard per il processo di sviluppo del software del computer e le relative funzioni di gestione aziendale.


\subsection{Staging}
\index{Staging}
Lo Staging è una fase del ciclo di vita del software nella quale si assemblano tutti i componenti e si testa il corretto funzionamento del sistema.

\subsection{Stakeholder}
\index{Stakeholder}
Uno Stakeholder è un soggetto, ente o persona,  direttamente o indirettamente coinvolti in un progetto o nell'attività di un'azienda.


\subsection{Stream processing}
\index{Stream processing}
Stream processing è un paradigma di programmazione parallela che permette ad alcune applicazioni di sfruttare semplicemente una forma limitata di elaborazione parallela.

\subsection{Strumenti di versionamento}
\index{Strumenti di versionamento}
Uno strumento di versionamento è un sistema che registra, nel tempo, i cambiamenti ad un file o ad una serie di file, così da poter richiamare una specifica versione in un secondo momento. 

\subsection{Stub}
\index{Stub}
Una porzione di codice utilizzata in sostituzione di altre funzionalità software in quanto può simulare il comportamento di codice esistente.

\subsection{Swift}
\index{Swift}
Swift è liguaggio di programmazione potente e intuitivo per macOS e iOS.

