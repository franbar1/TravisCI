\section{\quad$D\quad$}
\subsection{DAPPS}
\index{DAPPS}
DAPPS è un acronimo di decentralized application (applicazione decentralizzata), che consiste in una applicazione eseguita su una rete P2P (peer-to-peer) di computer, anziché su un singolo computer.

\subsection{Database}
\index{Database}
Un Database, ovvero una base di dati, è un archivio in cui le informazioni sono strutturate e collegatetra loro secondo un particolare modello logico (relazionale, gerarchico, a oggetti, ecc), offrendo una gestione effciente dei dati contenuti. 

\subsection{DevOps}
\index{DevOps}
DevOps è un approccio allo sviluppo del software che si basa sulla comunicazione, la collaborazione e l’integrazione tra il team di sviluppo e quello di operations (Dev+Ops), da qui il nome. La metodologia DevOps prevede un lavoro sinergico tra le due divisioni, quella degli sviluppatori, che si occupano della creazione del software e della scrittura del codice, e quella dei sistemisti, che gestiscono gli aspetti infrastrutturali e si occupano del mantenimento e della stabilità del software.

\subsection{Diagrammi di Gantt}
\index{Diagrammi di Gantt}
Il Diagramma di Gantt è uno strumento di supporto alla gestione dei progetti.


\subsection{Docker}
\index{Docker}
Docker è un progetto open source per automatizzare la distribuzione di app come contenitori portabili e autosufficienti che possono essere eseguiti nel cloud o in locale.

\subsection{Driver}
\index{Drive}
Codice fittizio che chima i stub. Viene utilizzato quando i sottomoduli sono pronti ma il modulo principale non è ancora pronto.

\subsection{DynamoDB}
\index{DynamoDB}
DynamoDB è database NoSQL ideato da Amazon.com. Tale Db supporta i modelli di dati di tipo documento e di tipo chiave-valore che offre prestazioni di pochi millisecondi a qualsiasi livello.


