\section{Voice Dialog Flow}
Il Voice Dialog Flow presenta degli esempi di comunicazione fra l’utente e l’assistente vocale Amazon Alexa. 
Durate le fasi successive di progettazione, le interazione possono essere soggette a cambiamenti o modifiche, in base alle richieste 
del proponente e/o committente. Sì identifica nel seguente modo i componenti di ogni iterazione:

\begin{itemize}
    \item \alexa rappresenta Amazon Alexa;
    \item \utente rappresenta l'utente che sta interagendo con Amazon Alexa;
    \item \textbf{[risorsa]:} il dato che viene prelevato al difuori del sistema dell'assistente vocale Amazon Alexa.
\end{itemize}
\subsection{Comandi Intent}
I comandi intent rappresentano un'azione che soddisfa la richiesta vocale dell'utente, questo per semplificare l'interazione e l'esecuzione
della \glo{skill} e dei workflow.\\

%------COMANDO STOP--------
\subsubsection{Comando Stop}
Comando che permette di fermare l'esecuzione di una skill oppure di un workflow.

\paragraph{Italiano}
\begin{itemize}
        \item \voice{stop};
        \item \voice{spegni};
        \item \voice{ferma}.
\end{itemize}

\paragraph{Inglese}
\begin{itemize}
        \item \voice{stop};
        \item \voice{turn off}.
\end{itemize}
%-----COMANDO SI ------
\subsubsection{Comando Sì}
Comando che permette di dare risposta affermativa alle domande poste da Amazon Alexa.

\paragraph{Italiano}
\begin{itemize}
        \item \voice{sì};
        \item \voice{perfetto};
        \item \voice{va bene}.
\end{itemize}

\paragraph{Inglese}
\begin{itemize}
        \item \voice{yes};
        \item \voice{all right};
        \item \voice{perfect};
        \item  \voice{of course}.
\end{itemize}

%-----COMANDO NO ------
\subsubsection{Comando No}
Comando che permette di dare risposta negativa alle domande poste da Amazon Alexa.

\paragraph{Italiano}
\begin{itemize}
        \item \voice{no};
        \item \voice{non va bene}.
\end{itemize}

\paragraph{Inglese}
\begin{itemize}
        \item \voice{no};
        \item \voice{not good}.
\end{itemize}


% Comando prossimo ( NEXT )
\subsubsection{Comando Next}
Comando che ti permetti di passare da un blocco al successivo di un workflow in un qualsiasi momento.
\paragraph{Italiano}
\begin{itemize}
        \item \voice{vai al prossimo};
        \item \voice{successivo};
        \item \voice{salta al prossimo};
        \item \voice{avanti}.
\end{itemize}

\paragraph{Inglese}
\begin{itemize}
        \item \voice{next};
        \item \voice{go ahead};
        \item \voice{jump to the next}.
\end{itemize}

% ---- Dialogo in Italiano ----- 
\subsection{Italiano}

%selezione skill
\subsubsection{Selezione Skill}

\utente \voice{Apri MegaAlexa} $|$ \voice{Esegui MegaAlexa} $|$ \voice{Avvia MegaAlexa} $|$ \voice{Fai partire MegaAlexa}.\\
\alexa  \voice{Benvenuto, ti trovi nella skill MegaAlexa. Come ti posso aiutare?} $|$ \voice{Con quale workflow partiamo?} $|$ \voice{Quale workflow desideri?}

%selezione workflow
\subsubsection{Selezione workflow}
Una volta che sì entra nella skill, ci sono due possibilità:

\begin{itemize}
    \item \utente \voice{Quali workflow sono presenti?} $|$ \voice{Mi sono dimenticato, mi elenchi i workflow.} $|$ \voice{Quali workflow posso utilizzare?}.\\
           \alexa  \voice{Sono presenti i seguenti workflow: [workflow]. Quali scegli?} $|$ \voice{Puoi scegliere tra i seguenti workflow: [workflow].}\\
           \utente  \voice{Eegui [workflow]} $|$ \voice{Avvia [workflow]} $|$ \voice{Apri [workflow]} $|$ \textit{Fai partire [workflow]}.
           %capire se eliminare il secondo utente
    \item \utente \voice{Apri [workflow]} $|$ \voice{Eegui [workflow]} $|$ \voice{Avvia [workflow]} $|$ \voice{Fai partire [workflow]}.\\
\end{itemize}


%Esecuzione del workflow.
\subsubsection{Esecuzione workflow}

\myparagraph{Testo}
\\
\alexa \voice{[testo]}

\myparagraph{Feed RSS}
\\
\alexa \voice{Gli aggiornamenti dei siti seguiti sono i seguenti: [feed rss]} $|$ 
       \voice{Ecco gli aggiornamenti dei siti seguiti: [feed rss]}

\myparagraph{Instagram}
\begin{itemize}
    \item Lettura delle notifiche:
     \begin{itemize}
        \item \alexa \voice{ Ecco le notifiche: [lettura notifiche]. Vuoi risentirle? } $|$ \voice{Queste sono le notifiche: [lettura notifiche]. Vuoi risentirle?}\\
              \utente \voice{[Comando no]}.\\
              \alexa \voice{Ok} 
        \item \alexa \voice{ Ecco le notifiche: [lettura notifiche]. Vuoi risentirle? } $|$ \voice{Queste sono le notifiche: [lettura notifiche]. Vuoi risentirle?}\\
                \utente \voice{[Comando sì]}.\\
                \alexa \voice{Va bene}
    \end{itemize}
    \item Visualizzazione foto da hashtag:
    \begin{itemize}
        \item \alexa \voice{ Ecco le foto selezionate: [visualizzazione foto].} $|$ \voice{Queste sono le foto: [visualizzazione foto].}
    \end{itemize}
\end{itemize}
Durante la configurazione del workflow, l'utente stabilisce, tramite il blocco di filtro, il numero di notifiche da leggere e per la visualizzazione delle foto
seleziona l'hashtag desiderato.

\myparagraph{Facebook}
\begin{itemize}
    \item Lettura delle notifiche:
     \begin{itemize}
        \item \alexa \voice{ Ecco le notifiche: [lettura notifiche].} $|$ \voice{Queste sono le notifiche ancora da leggere: [lettura notifiche].}\\
              \utente \voice{Va bene} $|$ \voice{Grazie}.
    \end{itemize}
    \item Lettura dei post:
    \begin{itemize}
        \item \alexa \voice{Eccoti gli ultimi post: [lettura post].} $|$ \voice{Ci siamo! ecco i posti i post: [lettura post].}\\
                \utente \voice{Va bene} $|$ \voice{Grazie}.  
    \end{itemize}
\end{itemize}

%LinkedIn
\myparagraph{LinkedIn}
\begin{itemize}
    \item Lettura delle notifiche:
     \begin{itemize}
        \item \alexa \voice{ Ecco le notifiche: [lettura notifiche].} $|$ \voice{Queste sono le notifiche ancora da leggere: [lettura notifiche].}\\
              \utente \voice{Va bene} $|$ \voice{Grazie}.
    \end{itemize}
    \item Lettura messaggi:
    \begin{itemize}
        \item Non ci sono messaggi:
        \begin{itemize}
            \item \alexa \voice{Mi dispiace, non ci sono messaggi da leggere.} $|$ \voice{Non hai messaggi da leggere.}
        \end{itemize}
        \item Ci sono messaggi:
        \begin{itemize}
            \item \alexa \voice{ Hai [n] messaggi da leggere, [lettura messaggi]. Leggo le sequenti?} $|$ \voice{Ci sono [n] messaggi da leggere, [lettura messaggi].}\\
            \utente \voice{Va bene} $|$ \voice{Grazie}.
     \item  \alexa \voice{ Hai [n] messaggi da leggere, [lettura messaggi].} $|$ \voice{ Ci sono [n] messaggi da leggere, [lettura messaggi].}\\
            \utente \voice{Va bene} $|$ \voice{Grazie}.
        \end{itemize}
    \end{itemize}
\end{itemize}

%Telegram

\myparagraph{Telegram}
\begin{itemize}
    \item Non ci sono messaggi da leggere:
    \begin{itemize}
        \item \alexa \voice{Relax! non hai messaggi non letti.}$|$ \voice{Non ci sono messaggi da leggere.}
    \end{itemize}
    \item Ci sono messaggi da leggere:
    \begin{itemize}
        \item \alexa \voice{ Hai [n] messaggi da leggere, [lettura messaggi].}$|$ \voice{ Ci sono [n] messaggi da leggere, [lettura messaggi].}\\
                \utente \voice{Va bene} $|$ \voice{Grazie}.
    \end{itemize}
    
\end{itemize}


%Slack
\myparagraph{Slack}
\begin{itemize}
    \item Non ci sono messaggi da leggere:
    \begin{itemize}
        \item \alexa \voice{Relax! non hai messaggi non letti.}$|$ \voice{Non ci sono messaggi da leggere.}
    \end{itemize}
    \item Ci sono messaggi da leggere:
    \begin{itemize}
        \item \alexa \voice{ Hai [n] messaggi da leggere, [lettura messaggi].}$|$ \voice{ Ci sono [n] messaggi da leggere, [lettura messaggi].}\\
                \utente \voice{Va bene} $|$ \voice{Grazie}.
    \end{itemize}
    
\end{itemize}


%Messanger
\myparagraph{Messanger}
\begin{itemize}
    \item Non ci sono messaggi da leggere:
    \begin{itemize}
        \item \alexa \voice{Relax! non hai messaggi non letti.}$|$ \voice{Non ci sono messaggi da leggere.}
    \end{itemize}
    \item Ci sono messaggi da leggere:
    \begin{itemize}
        \item \alexa \voice{ Hai [n] messaggi da leggere, [lettura messaggi].}$|$ \voice{ Ci sono [n] messaggi da leggere, [lettura messaggi].}\\
                \utente \voice{Va bene} $|$ \voice{Grazie}.
    \end{itemize}
    
\end{itemize}


\myparagraph{Meteo}
\\
\alexa \voice{ A [località scelta] ci sono [gradi]} $|$ \voice{ In questo momento ci sono [gradi] a [località scelta]}

\myparagraph{Mail}
\begin{itemize}
    \item Ci sono email da leggere:
    \begin{itemize}
        \item \alexa \voice{ Hai [n] email da leggere, [emails]. Leggo le sequenti?} $|$ \voice{ Ci sono [n] email da leggere, [emails]. Leggo le restanti?}\\
               \utente \voice{[Comando sì]}
        \item \alexa \voice{ Hai [n] email da leggere, [emails]. Leggo le sequenti?} $|$ \voice{ Ci sono [n] email da leggere, [emails]. Leggo le restanti?}\\
                \utente \voice{[Comando no]}
    \end{itemize}
    \item Non ci sono email da leggere:
    \begin{itemize}
        \item \alexa \voice{Non ci sono email da leggere.} $|$ \voice{Non hai email da leggere.}
    \end{itemize}
\end{itemize}
Il numero di email che verranno lette dall’assistente vocale Alexa, dipenderà dal blocco filtro impostato dall'utente. 
Qualora non sarà presente nessun filtro , Alexa leggerà tutte le nuove mail presenti.

\myparagraph{Calendario}
\begin{itemize}
    \item Lettura eventi:
    \begin{itemize}
        \item L'utente non ha nessun evento in programma:
        \\\alexa \voice{Non ci sono eventi registrati per oggi} | \voice{Relax, non ci sono eventi oggi}
        \\ \utente \voice{Ok} $|$ \voice{Grazie} $|$ \voice{Perfetto}
        \item L'utente ha eventi:
        \\ \alexa \voice{Oggi hai i seguenti eventi: [eventi]} $|$ \voice{Oggi sei impegnato coi sequenti eventi: [eventi]}
        \\ \utente \voice{Ok} $|$ \voice{Perfetto} $|$ \voice{Grazie}
    \end{itemize}
    \item Scrittura eventi:
     \\\alexa \voice{Che nome vuoi dare all'evento?}
            \\ \utente \voice{[nome evento]}
            \\ \alexa \voice{Qual è il giorno?} $|$ \voice{In quale giorno sì avverrà?}
            \\ \utente \voice{[giorno evento]}
            \\ \alexa \voice{A che ora?}  $|$ \voice{A che ora sì terrà?}
            \\ \utente \voice{[ora evento]}
\end{itemize}


%Youtube 
\myparagraph{YouTube}
\begin{itemize}
    \item Non ci sono nuovi video:
    \begin{itemize}
        \item \alexa \voice{Non sono presenti nuovi video nel canale prescelto.}
    \end{itemize}
    \item Ci sono nuovi video nel canale:
    \begin{itemize}
        \item \alexa \voice{Questo è l'ultimo video caricato nel canale prescelto [link video].}$|$ \alexa \voice{Ultimo video caricato del canale prescelto [link video].}
        \utente \voice{Va bene} $|$ \voice{Grazie}.
    \end{itemize}
\end{itemize}

\myparagraph{YouTube Music}
\\
\alexa \voice{Eccoti la tua playlist, buon ascolto [playlist].} $|$ \voice{Buon ascolto [playlist].}

\myparagraph{Spotify}
\\
\alexa \voice{Ecco la musica scelta per te [playlist].} $|$ \voice{Relassati e buon ascolto [playlist].}


\myparagraph{Radio}
\\
\alexa \voice{Ci siamo , ecco la tua radio preferita, buon ascolto: [radio]}

\myparagraph{Programmazione TV}
\\
\alexa \voice{I programmi televisivi di oggi sono: [programmi TV]}$|$ \voice{Questi sono i programmi di oggi: [programmi TV].}

\myparagraph{Cinema}
\\
\alexa \voice{Oggi al cinema ci sono le seguenti proiezioni : [film]}$|$ \voice{Questi sono i fiml in proiezione oggi: [film].}
%TRASPORTI
\myparagraph{Trasporti}
\\
\alexa \voice{Gli orari del [trasporto] sono i seguenti: [orari]}
\\ \\
L'utente durante la configurazione del blocco Trasporti, stabilisce la partenza e l'arrivo.
%SICUREZZA
\myparagraph{Sicurezza}
\\ \\
\alexa \voice{Qual è il tuo codice di sicurezza?}$|$ \voice{Dimmi il tuo codice di sicurezza}$|$ \voice{Qual è il tuo codice?}\\
\utente \voice{[codice di sicurezza]}.\\
L'operazione va avanti finché l'utente non dirà il codice di sicurezza corretto. Una volta detto corretto, sì procede col workflow.

%LISTA 
\myparagraph{Lista}
\begin{itemize}
    \item La lista è vuota:
    \begin{itemize}
        \item Aggiunta di un elemento alla lista:\\
        \alexa\voice{Che elemento vuoi aggiungere alla lista?} $|$ \voice{La lista è vuota, che elemento vuoi aggiungere?}\\
        \utente \voice{Vorrei aggiungere [elemento] alla lista} $|$ \voice{Voglio aggiungere [elemento] alla lista} $|$
        \voice{Aggiungi [elemento] alla lista.}\\
        \alexa \voice{Aggiunto, vuoi aggiungere altro? } $|$ \voice{Elemento aggiunto con successo, vuoi aggiungere altro?} $|$ \voice{Fatto, desideri aggiungere altro?}\\
        \utente \voice{[comando no]}
    \end{itemize}
    \item La lista non è vuota:
    \begin{itemize}
        \item Aggiunta di un elemento alla lista:
        \\ \alexa\voice{Sono presenti questi elementi nella lista: [elementi]. Cosa vuoi aggiungere?}$|$ \voice{Gli elementi presenti nella lista in questo momento sono: [elementi]. Cosa vuoi aggiungere?}\\
        \utente \voice{Vorrei aggiungere [elemento] alla lista} $|$ \voice{Voglio aggiungere [elemento] alla lista} $|$
        \voice{Aggiungi [elemento] alla lista.}\\
    \end{itemize}
\end{itemize}

\subsubsection{Conclusione workflow}
\alexa \voice{La creazione del workflow è andata a buon fine.}

%%%%%%%%%%%%%%%%%%%%%%%%%%%%%%%%%%%%%%%%%%%%%%%%%%%%%%%%%%%%%%%%%%%%%%%%%%%%%%%%%%%%%%%%%%%%%%%%%%%%%%%%%%%%%%%%%%%%%%%%%%%%%%%%%%%%%%%%%%%%%%%%%%%%%%%%%%%%%%%%%%%%%%%%%%%%%%%%%%%%%%%%%%%%%%%

\subsection{English}

%selezione skill
\subsubsection{Skill selection}

\utente \voice{Open MegaAlexa} $|$ \voice{Execute MegaAlexa} $|$ \voice{Start MegaAlexa} $|$ \voice{Switch on MegaAlexa}\\
\alexa  \voice{Welcome, you are using the MegaAlexa skill. How can I help you?} $|$ \voice{What workflow do we start with?} $|$ \voice{Which workflow do you want?}

%selezione workflow
\subsubsection{Workflow selection}
Once you get into the skill, there are two possibilities:

\begin{itemize}
    \item \utente \voice{Which workflows are available?} $|$ \voice{I forgot, please repeat the workflows?} $|$ \voice{Which workflows can I use?}.\\
           \alexa  \voice{ The workflows available are as follow: [workflow]. Which one do you choose?} $|$ \voice{You can choose the following workflows: [workflow].}\\
           \utente  \voice{Execute [workflow]} $|$ \voice{Start [workflow]} $|$ \voice{Open [workflow]}.
           %capire se eliminare il secondo utente
    \item \utente \voice{Open [workflow]} $|$ \voice{Execute [workflow]} $|$ \voice{Start [workflow]}.\\
\end{itemize}


%Esecuzione del workflow.
\subsubsection{Workflow execution}

\myparagraph{Text}
\\
\alexa \voice{[text]}

\myparagraph{Feed RSS}
\\
\alexa \voice{The updates of the followed sites are: [feed rss]} $|$ 
       \voice{Here you have the updates of the followed sites: [feed rss]}

\myparagraph{Instagram}
\begin{itemize}
    \item Reading notifications:
     \begin{itemize}
        \item \alexa \voice{ Here's the notifications: [reading notifications]. Do you want to hear them again? } $|$ \voice{These are notifications: [reading notifications]. Do you want to hear them again?}\\
              \utente \voice{[No command]}.\\
              \alexa \voice{Ok} 
        \item \alexa \voice{ Here's the notifications: [reading notifications]. Do I have to repeat them? } $|$ \voice{These are notifications: [reading notifications]. Do I have to repeat them?}\\
                \utente \voice{[Yes command]}.\\
                \alexa \voice{Ok}
    \end{itemize}
    \item Viewing photos from hashtags:
    \begin{itemize}
        \item \alexa \voice{ Here you have the selected pictures: [display photos].} $|$ \voice{These are the pictures: [display pictures].}
    \end{itemize}
\end{itemize}
During the configuration of the workflow, by using the filter block, the user sets how many notifications to read and visualise.

\myparagraph{Facebook}
\begin{itemize}
    \item Reading notifications:
     \begin{itemize}
        \item \alexa \voice{ Here are the notifications: [reading notifications].} $|$ \voice{These are the notifications you haven’t read yet: [reading notifications].}\\
              \utente \voice{Ok} $|$ \voice{Thank you}.
    \end{itemize}
    \item Reading posts:
    \begin{itemize}
        \item \alexa \voice{Here are the latest posts: [reading posts].} $|$ \voice{Here we are! here are the posts: [reading posts].}\\
                \utente \voice{Ok} $|$ \voice{Thank you}.  
    \end{itemize}
\end{itemize}

%LinkedIn
\myparagraph{LinkedIn}
\begin{itemize}
    \item Reading notifications:
     \begin{itemize}
        \item \alexa \voice{ Here's the notifications: [reading notifications].} $|$ \voice{These are the notifications yet to be read: [reading notifications].}\\
              \utente \voice{Ok} $|$ \voice{Thank you}.
    \end{itemize}
    \item Reading messages:
    \begin{itemize}
        \item There are no messages:
        \begin{itemize}
            \item \alexa \voice{I'm sorry, there are no messages to read.} $|$ \voice{There are no messages to read.}
        \end{itemize}
        \item There are messages:
        \begin{itemize}
            \item \alexa \voice{  You have [n] new messages, [reading messages]. Sould I read the rest of them?} $|$ \voice{There are [n] messages to read, [reading messages].}\\
            \utente \voice{Ok} $|$ \voice{Thank you}.
     \item  \alexa \voice{ You have [n] messages to read, [reading messages].} $|$ \voice{ There are [n] messages to read, [reading messages].}\\
            \utente \voice{Ok} $|$ \voice{Thank you}.
        \end{itemize}
    \end{itemize}
\end{itemize}

%Telegram

\myparagraph{Telegram}
\begin{itemize}
    \item There are no messages to read:
    \begin{itemize}
        \item \alexa \voice{Relax! there are no new messages.}$|$ \voice{There are no messages to read.}
    \end{itemize}
    \item There are messages to read:
    \begin{itemize}
        \item \alexa \voice{ You have [n] messages to read, [reading messages].}$|$ \voice{ There are [n] messages to read, [reading messages].}\\
                \utente \voice{Ok} $|$ \voice{Thank you}.
    \end{itemize}
    
\end{itemize}


%Slack
\myparagraph{Slack}
\begin{itemize}
    \item There are no messages to read:
    \begin{itemize}
        \item \alexa \voice{Relax! there are no new messages.}$|$ \voice{There are no messages to read.}
    \end{itemize}
    \item There are messages to read:
    \begin{itemize}
        \item \alexa \voice{ You have [n] messages to read, [reading messages].}$|$ \voice{ There are [n] messages to read, [reading messages].}\\
                \utente \voice{Ok} $|$ \voice{Thank you}.
    \end{itemize}
    
\end{itemize}


%Messanger
\myparagraph{Messanger}
\begin{itemize}
    \item There are no messages to read:
    \begin{itemize}
        \item \alexa \voice{Relax! there are no new messages.}$|$ \voice{There are no messages to read.}
    \end{itemize}
    \item There are messages to read:
    \begin{itemize}
        \item \alexa \voice{ You have [n] messages to read, [reading messages].}$|$ \voice{ There are [n] messages to read, [reading messages].}\\
                \utente \voice{Ok} $|$ \voice{Thank you}.
    \end{itemize}
    
\end{itemize}


\myparagraph{Weather}
\\
\alexa \voice{ In [chosen location] there are [degree]} $|$ \voice{ In this moment there are [degree] in [chosen location]}

\myparagraph{Mail}
\begin{itemize}
    \item There are emails to read:
    \begin{itemize}
        \item \alexa \voice{ You have [n] emails to read, [emails]. Do I have to read them} $|$ \voice{ There are [n] emails to read, [emails]. Do I read the rest?}\\
               \utente \voice{[Yes command]}
        \item \alexa \voice{ You have [n] emails to read, [emails]. Do I have to read them?} $|$ \voice{ There are [n] emails to read, [emails]. Do I read the rest?}\\
                \utente \voice{[No command]}
    \end{itemize}
    \item There are no emails to read:
    \begin{itemize}
        \item \alexa \voice{There are emails to read.} $|$ \voice{There are no emails to read.}
    \end{itemize}
\end{itemize}
The number of emails that alexa will read, depends on the filter block imposted by the user. If no filter are imposted, alexa will read all the emails.

\myparagraph{Calendar}
\begin{itemize}
    \item Reading events:
    \begin{itemize}
        \item The user does not have any scheduled events:
        \\\alexa \voice{There are no events today} | \voice{Relax, there are no events today}
        \\ \utente \voice{Ok} $|$ \voice{Thank you} $|$ \voice{Perfect}
        \item The user has events:
        \\ \alexa \voice{Today you have the following lineup: [events]} $|$ \voice{Today you're busy with the following events: [events]}
        \\ \utente \voice{Ok} $|$ \voice{Perfect} $|$ \voice{Thank you}
    \end{itemize}
    \item Writing events:
     \\\alexa \voice{What is the name of the event?}
            \\ \utente \voice{[event name]}
            \\ \alexa \voice{Which day?} $|$ \voice{On which day will it happen?}
            \\ \utente \voice{[event day]}
            \\ \alexa \voice{At which time?}  $|$ \voice{Which time will it be held??}
            \\ \utente \voice{[event time]}
\end{itemize}

%Youtube 
\myparagraph{YouTube}
\begin{itemize}
    \item There are no new videos:
    \begin{itemize}
        \item \alexa \voice{There are no new videos in the chosen channel.}
    \end{itemize}
    \item There are new videos in the channel:
    \begin{itemize}
        \item \alexa \voice{This is the last video uploaded in the chosen channel [link video].}$|$ \alexa \voice{Last video uploaded of the chosen channel [link video].}
        \utente \voice{Ok} $|$ \voice{Thank you}.
    \end{itemize}
\end{itemize}

\myparagraph{YouTube Music}
\\
\alexa \voice{Here's the playlist, enjoy [playlist].} $|$ \voice{Enjoy [playlist].}

\myparagraph{Spotify}
\\
\alexa \voice{Here's the music chosen for you [playlist].} $|$ \voice{Relax and enjoy [playlist].}


\myparagraph{Radio}
\\
\alexa \voice{Here we go, here's your favorite radio, enjoy: [radio]}

\myparagraph{TV Programming}
\\
\alexa \voice{TV programs of today are: [TV programs]}$|$ \voice{These are the programs of today: [TV programs].}

\myparagraph{Movie theater}
\\
\alexa \voice{ These are the movies on today: [movies]}$|$ \voice{There are the movies in the theater today: [movies].}
%TRASPORTI
\myparagraph{Trasportation}
\\
\alexa \voice{The timetables of the [trasportation] are the following: []}
\\ \\
During the configuration of the transportation block , the user sets the starting location and the destination.
%SICUREZZA
\myparagraph{Security}
\\ \\
\alexa \voice{What is your security code?}$|$ \voice{Tell me your security code}$|$ \voice{What is your code?}\\
\utente \voice{[security code]}.\\
The operation continues until the user pronounces the correct password. Once pronounced, the workflow can initiate.
%LISTA 
\myparagraph{List}
\begin{itemize}
    \item The list is empty:
    \begin{itemize}
        \item Adding an item to the list:\\
        \alexa\voice{What item do you want to add to the list?} $|$ \voice{The list is empty, what item do you want to add?}\\
        \utente \voice{I would like to add [item] to the list} $|$ \voice{I want to add [item] to the list} $|$
        \voice{Add [item] to the list.}\\
        \alexa \voice{The item has been successfully added. Do you want to add anything alse? } $|$ \voice{Done. Do you want to add anything else?} $|$ \voice{Done, would you like to add more?}\\
        \utente \voice{[No command]}
    \end{itemize}
    \item The list is not empty:
    \begin{itemize}
        \item Adding an item to the list:
        \\ \alexa\voice{These items are on the list: [items]. What do you want to add?}$|$ \voice{At the moment the items on the list are the following: [items]. Do you want to add more?}\\
        \utente \voice{I would like to add [item] to the list} $|$ \voice{I want to add [item] to the list} $|$
        \voice{Add [item] to the list.}\\
    \end{itemize}
\end{itemize}

\subsubsection{Workflow ending}
\alexa \voice{Workflow creation was successful.}