\appendix
\section{Riscontri dei rischi}
\begin{center}
    \renewcommand{\arraystretch}{2.2}
    \rowcolors{3}{tableRow}{}
    
    \begin{longtable}{ p{0.1\textwidth} p{0.2\textwidth} p{0.2\textwidth} p{0.2\textwidth}}

        \rowcolor[HTML]{232f3e}
    
        \rowcolors{3}{tableRow}{}
        \color[HTML]{FFFFFF} \textbf{ID} & \color[HTML]{FFFFFF} \centering\textbf{Periodo} & \color[HTML]{FFFFFF} \textbf{Scenario} & \color[HTML]{FFFFFF} \textbf{Manutenzione migliorativa}\\
    \endhead
    \rowcolor{white}\multicolumn{3}{c}
    { Continua nella pagina successiva} \\
    \endfoot
    \caption{Riscontro dei rischi}
        \label{tab:risc}
    \endlastfoot
    RT1 & Stesura dei documenti e condivisione di una repository & Questo rischio riguarda le difficoltà di apprendimento di Latex e GitLab, in quanto qualche membro del gruppo non avevano familiarità con queste tecnologie & Coloro che non conoscevano queste tecnologie hanno dovuto dedicare del tempo per l'apprendimento e chi già le conosceva ha dedicato del tempo nell'insegnamento\\
    
    RG1 & Rischio costante & Alcuni membri del gruppo sono dei lavoratori e questo fattore ha rallentato il gruppo. Questo rischio aumenta la probabilità di occorrenza del rischio RG2 & Chiunque avesse impedimenti di qualsiasi genere è riuscito a svolgere il compito assegnatogli dal \textit{responsabile}. Ad ogni modo, essendo tutti i membri del gruppo a conoscenza di queste problematiche il gruppo si prefissa l'obiettivo di andare incontro a chi avesse impegni improrogabili aiutandoli nei loro incarichi\\
    
    RG2 & Piano di qualifica & I verificatori ritengono il documento prodotto insufficiente e con lacune a ridosso della scadenza prevista per l'RR dovendo in poco tempo fare importanti modifiche al documento poiché chi incaricato della sua stesura è impossibilitato a causa del rischio RG1 di rivedere quanto scritto & Non ritrovarsi a ridosso della consegna in queste situazioni. Questo rischio deriva in parte dal RG3. Il gruppo deve quindi imparare da questa situazioni affinché non ricapiti in futuro organizzando la distribuzione del lavoro in modo da coprire eventuali situazioni analoghe \\

    RG3 & Rischio costante & Situazione per cui ci si trova in difficoltà con la consegna costringendo i membri del gruppo a un carico di lavoro maggiore rispetto a quanto previsto & Questo rischio aumenta la probabilità di occorrenza di RG2. Le contromisure adottate saranno quelle di un'organizzazione del lavoro maggiormente accurata \\
    
    RO2 & 29-03-2019 & Tutte le aule sono occupate da altre attività didattiche & Il gruppo deve utilizzare i mezzi a sua disposizione per conoscere in anticipo (il giorno prima almeno) le aule libere dove poter lavorare insieme \\
    
    RR3 & Analisi dei requisiti & Le domande rivolte all'azienda \proposerName{} non si sono rivelate sempre esaustive lasciando aperti alcuni dubbi riguardo ai casi d'uso dell'analisi dei requisiti & Nel caso di dubbi particolarmente importanti non risolti, il gruppo si impegna di insistere con l'azienda proponente. Qualora il dubbio non venga risolto il gruppo deve essere abile nell'adattarsi alle risposte ricevute prendendo una decisione definitiva e senza restare nell'incertezza \\
    
    RS1 & Piano di qualifica & La stesura di questo documento ha causato ritardi e difficoltà non rispettando le scadenze temporali prefissate & I membri del gruppo devono impegnarsi affinché ciò non accada di nuovo facendo esperienza di situazioni come questa. Il gruppo deve capire in anticipo la possibilità di incontrare questo rischio e di conseguenza redistribuire il carico di lavoro in modo da rispettare le scadenze\\
    
    
    \end{longtable}
    
    \end{center}

    
    \section{Organigramma}
        \subsection{Redazione}
            \begin{table}[H]
                \centering
                \renewcommand{\arraystretch}{2.8}
                \rowcolors{3}{tableRow}{}
                \begin{tabular}{c c c}
                    \rowcolor[HTML]{232f3e} 
                    \multicolumn{1}{c}{\color[HTML]{FFFFFF} \textbf{Nome}} &
                    \multicolumn{1}{c}{\color[HTML]{FFFFFF} \textbf{Data}} &
                    \multicolumn{1}{c}{\color[HTML]{FFFFFF} \textbf{Firma}} \\
                    
                    \andrea&17-03-2019&\includegraphics[width=0.2\textwidth]{immagini/firme/andrea_firma.png}\\
                \end{tabular}
                \caption{Redazione} \label{table:redazione}
            \end{table}
    
        \subsection{Approvazione}
            \begin{table}[H]
                \centering
                \renewcommand{\arraystretch}{2.8}
                \rowcolors{3}{tableRow}{}
                \begin{tabular}{c c c}
                    \rowcolor[HTML]{232f3e} 
                    \multicolumn{1}{c}{\color[HTML]{FFFFFF} \textbf{Nome}} &
                    \multicolumn{1}{c}{\color[HTML]{FFFFFF} \textbf{Data}} &
                    \multicolumn{1}{c}{\color[HTML]{FFFFFF} \textbf{Firma}} \\
                 
                    \valentin&17-03-2019&\includegraphics[width=0.2\textwidth]{immagini/firme/valentin_firma.png}\\
                    \commitNameM&&\\
                \end{tabular}
                \caption {Approvazione} \label{table:approvazione}
            \end{table}
    
        \subsection{Accettazione dei componenti}
            \begin{table}[H]
            \centering
            \renewcommand{\arraystretch}{2.8}
            \rowcolors{3}{tableRow}{}
                \begin{tabular}{c c p{3.5cm}}
                    \rowcolor[HTML]{232f3e} 
                    \multicolumn{1}{c}{\color[HTML]{FFFFFF} \textbf{Nome}} &
                    \multicolumn{1}{c}{\color[HTML]{FFFFFF} \textbf{Data}} &
                    \multicolumn{1}{c}{\color[HTML]{FFFFFF} \textbf{Firma}} \\
                    
                    \daniele&18-03-2019&\includegraphics[width=0.2\textwidth]{immagini/firme/daniele_firma.png}\\
                    \giacomo&18-03-2019&\includegraphics[width=0.2\textwidth]{immagini/firme/giacomo_firma.png}\\
                    \francesco&18-03-2019&\includegraphics[width=0.2\textwidth]{immagini/firme/francesco_firma.png}\\
                    \davide&18-03-2019&\includegraphics[width=0.2\textwidth]{immagini/firme/davide_firma.png}\\
                    \valentin&18-03-2019&\includegraphics[width=0.2\textwidth]{immagini/firme/valentin_firma.png}\\
                    \andrea&18-03-2019&\includegraphics[width=0.2\textwidth]{immagini/firme/andrea_firma.png}\\
                    \singh&18-03-2019&\includegraphics[width=0.2\textwidth]{immagini/firme/singh_firma.png}\\
                \end{tabular}
                \caption {Accettazione dei componenti} \label{table:accettazione dei componenti}
            \end{table}
    
        \subsection{Componenti}
            \begin{table}[H]
            \centering
            \renewcommand{\arraystretch}{2.8}
            \rowcolors{3}{tableRow}{}
                \begin{tabular}{c c c}
                    \rowcolor[HTML]{232f3e} 
                    \multicolumn{1}{c}{\color[HTML]{FFFFFF} \textbf{Nome}} &
                    \multicolumn{1}{c}{\color[HTML]{FFFFFF} \textbf{Matricola}} &
                    \multicolumn{1}{c}{\color[HTML]{FFFFFF} \textbf{Email}} \\
                    
                    \daniele&1127165&daniele.scialabba@studenti.unipd.it\\
                    \giacomo&1122451&giacomo.corro@studenti.unipd.it\\
                    \francesco&1122214&francesco.barbanti@studenti.unipd.it\\
                    \davide&1122285&davide.tognon.3@studenti.unipd.it\\
                    \valentin&1099561&valentin.grigoras@studenti.unipd.it\\
                    \andrea&1143621&andrea.chinello.5@studenti.unipd.it\\
                    \singh&1102486&sukhjinder.singh@studenti.unipd.it\\
                \end{tabular}
                \caption {Componenti} \label{table:Componenti}
        \end{table}
        
    
    
