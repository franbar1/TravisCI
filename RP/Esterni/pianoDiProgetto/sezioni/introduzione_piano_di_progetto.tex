\justify
\section{Introduzione}

\subsection{Scopo del documento}
Questo documento descrive quali sono le risorse disponibili e le loro modalità di assegnamento a processi, attività e task
di cui ne mostra la scansione temporale. 
Lo scopo è organizzare le attività in maniera efficiente per ottenere soluzioni efficaci. Questo viene facilitato
da una scansione delle attività nel tempo.

\subsection{Struttura}
Il \docNamePdP{} è strutturato nel seguente modo:
\begin{itemize}
    \item Organizzazione del Progetto;
    \item Analisi dei rischi;
    \item Risorse disponibili;
    \item Suddivisione del lavoro;
    \item Calendario delle attività;
    \item Meccanismi di controllo e rendicontazione.
\end{itemize}

\subsection{Scopo del Prodotto}
La principale richiesta è rendere disponibile, all’utente registrato all'applicazione, delle micro-funzioni chiamate \glo{connettori}.
Tali connettori sono inseriti all’interno di un \glo{workflow} svolto da un controllo vocale. \\
Il progetto utilizzerà l’infrastruttura \glo{Amazon Web Services} in particolare le tecnologie \glo{Lambda}, \glo{API Gateway}, \glo{Aurora Serverless} e \glo{DynamoDB}. La comunicazione dei risultati all’utente avviene tramite l’assistente vocale \glo{Amazon Alexa}, l’applicazione mobile o web.

\subsection{Note esplicative}
Al fine di evitare ambiguità ai lettori non interni al gruppo, si specifica l'utilizzo di convenzioni prese da \groupName{} per la stesura dei documenti, che sono le seguenti: 
\begin{itemize}
\item \textbf{Glossario}: per evitare ridondanze e ambiguità di linguaggio e massimizzare la comprensione dei documenti, i termini tecnici, di dominio, e gli acronimi  che necessitano di una spiegazione, sono  definiti e descritti nel \docNameVersionGlo{}. I vocaboli riportati nel \docNameVersionGlo{} sono marcati da una "G" maiuscola a pedice;
\item \textbf{Documentazione}: i nomi degli altri documenti prodotti dal gruppo \groupName{} compariranno sempre in corsivo, seguiti da una "D" a pedice.
\end{itemize}

\subsection{Riferimenti}

\subsubsection{Normativi}
\begin{itemize}
    \item \textbf{Norme di Progetto}: \docNameVersionNdP{};
    \item \textbf{Capitolato C4 - MegAlexa}\\ \url{https://www.math.unipd.it/~tullio/IS-1/2018/Progetto/C4.pdf};
    \item \textbf{VER-E-22-03-2019} relativo all'incontro con l'azienda \proposerName{} avvenuto in data 22-03-2019 tramite conference call con \glo{Google Hangouts}.
\end{itemize}

\subsubsection{Informativi}
\begin{itemize}
        \item \textbf{Ingegneria del software}, Ian Sommerville, Pearson Education, Addison Wesley (Decima edizione)
        \begin{itemize}
            \item capitolo 19.1.
        \end{itemize}
        \item \textbf{Slide "Gestione di progetto"} nella pagina del corso:\\ \url{https://www.math.unipd.it/~tullio/IS-1/2018/Dispense/L06.pdf}
\end{itemize}

\subsection{Scadenze}
Tutti i componenti del gruppo hanno deciso di seguire le seguenti milestone per lo svolgimento del progetto:
\begin{itemize}
    \item \textbf{Revisione dei Requisiti (RR)}: 19-04-2019;
    \item \textbf{Revisione di Progettazione (RP)}: 17-05-2019;
    \item \textbf{Revisione di Qualifica (RQ)}: 17-06-2019;
    \item \textbf{Revisione di Accettazione (RA)}: 15-07-2019.
\end{itemize}
