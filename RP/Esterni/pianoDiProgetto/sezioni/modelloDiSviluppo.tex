\section{Modello di sviluppo}
La scelta di un modello di sviluppo porta dei vincoli sulla pianificazione e sulla gestione del progetto.
L'obiettivo del gruppo è di abbinare qualità al modello scelto, in modo da ottenere conformità con gli obiettivi del modello nel progetto
e maturità nei processi.\\ 
Il gruppo ha optato di scegliere il modello incrementale per il \glo{ciclo di vita} del software per i seguenti motivi:
\begin{itemize}
    \item Può produrre valore ad ogni incremento, aiutando a fissare meglio i requisiti per gli incrementi successivi;
    \item Ogni incremento riduce il rischio di fallimento;
    \item Le funzionalità principali sono sviluppate nei primi incrementi che diventano via via più stabili.
\end{itemize}
Nel modello incrementale i requisiti vengono classificati in base alla loro importanza strategica. In questo modo
quelli più importanti vengono trattati prima. Questo ne aumenta la chiarezza e la facilità di soddisfazione. I requisiti
meno importanti invece vengono soddisfatti dopo quelli più importanti venendo così inseriti in un sistema già stabilizzato. \\
Il metodo di lavoro sarà quindi questo:
\begin{itemize}
    \item In ogni fase di lavoro vengono prefissati degli incrementi che devono essere prodotti entro un scadenza decisa dal gruppo;
    \item Il lavoro viene diviso tra i membri del gruppo;
    \item Al termine del periodo prefissato una riunione servirà per analizzare il lavoro svolto da ogni membro, riscontrare problemi e difficoltà;
    \item Sarà compito dei \textit{verificatori} controllare il lavoro svolto dagli altri membri del gruppo e sollevare eventuali incongruenze o errori;
    \item Alla fine di questa \glo{verifica} seguirà una nuova discussione di gruppo per stabilire se gli obiettivi dell'incremento sono stati soddisfatti. 
\end{itemize}

\begin{figure}[H]
    \centering
    \includegraphics[scale=0.75]{immagini/incrementale.png}
    \caption{Modello incrementale}
\end{figure}