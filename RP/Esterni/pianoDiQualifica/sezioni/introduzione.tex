\justify
\section{Introduzione}
    \subsection{Scopo del documento}
        Lo scopo del Piano di Qualifica è documentare le strategie di \glo{verifica} e \glo{validazione} che il gruppo \groupName{} ha deciso
        di adottare per raggiungere gli obiettivi di qualità di processo e di prodotto relativi al progetto \projectName{}.
        A tal fine occorre verificare continuamente le attività svolte in modo da ottenere uno sviluppo del prodotto per costruzione e non per correzione minimizzando l'utilizzo delle risorse.  

    \subsection{Scopo del prodotto}
        Lo scopo del prodotto è quello di sviluppare un applicativo mobile che sia in grado di guidare l'utente, registrato, nella 
        creazione dei \glo{workflow}, cioè delle routine personalizzate. Una volta effettuata la creazione, l'utente tramite \glo{Amazon Alexa} può 
        eseguire le routine create quando desidera.
    \subsection{Note esplicative}
        Al fine di evitare ambiguità ai lettori non interni al gruppo, si specifica l'utilizzo di convenzioni prese da HexaDec per la stesura dei documenti, che sono le seguenti: 
       \begin{itemize}
        \item \textbf{Glossario}: per evitare ridondanze e ambiguità di linguaggio e massimizzare la comprensione dei documenti, i termini tecnici, di dominio, e gli acronimi  che necessitano di una spiegazione, sono  definiti e descritti nel \docNameVersionGlo. I vocaboli riportati nel \docNameVersionGlo{} sono marcati da una "G" maiuscola a pedice.
        \item \textbf{Documentazione}: i nomi degli altri documenti prodotti dal gruppo HexaDec compariranno sempre in corsivo, seguiti da una "D" a pedice.
        \end{itemize}

    \subsection{Riferimenti}
        \subsubsection{Riferimenti normativi}
        \begin{itemize}
            \item \textbf{Norme di Progetto}: \docNameVersionNdP;
            \item \textbf{Standard ISO/IEC 9126}: \url{https://it.wikipedia.org/wiki/ISO/IEC_9126};
            \item \textbf{Standard ISO/IEC 15504}: \url{https://en.wikipedia.org/wiki/ISO/IEC_15504};
            \item \textbf{PCDA: } \url{https://it.wikipedia.org/wiki/Ciclo_di_Deming}.
        \end{itemize}
    
        \subsubsection{Riferimenti Informativi}
            \begin{itemize}
                \item \textbf{Indice di Gulpese: } 
                \url{https://it.m.wikipedia.org/wiki/Indice_Gulpease};
                \item \textbf{Qualità di prodotto - Slide del corso di Ingegneria del Software: }
                
                \url{https://www.math.unipd.it/~tullio/IS-1/2018/Dispense/L13.pdf};
                \item \textbf{Qualità di processo - Slide del corso di Ingegneria del Software: }
                    
                \url{https://www.math.unipd.it/~tullio/IS-1/2018/Dispense/L14.pdf}
            \end{itemize} 
       
