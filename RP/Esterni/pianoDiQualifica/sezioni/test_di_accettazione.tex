\section{Test di accettazione}

\begin{center}
\renewcommand{\arraystretch}{2.2}
\rowcolors{3}{tableRow}{}

\begin{longtable}{ c m{12cm} c }
	
	\rowcolor[HTML]{232f3e}

	\rowcolors{3}{tableRow}{}
	\color[HTML]{FFFFFF} \textbf{Id Test} & \color[HTML]{FFFFFF} \centering\textbf{Descrizione} & \color[HTML]{FFFFFF} \textbf{Stato} \\
\endhead
\rowcolor{white}\multicolumn{2}{c}
   { Continua nella pagina successiva} \\
   \endfoot
   \caption [Test di validazione]{Test di validazione}
	\label{tabella:testVal}
   \endlastfoot
%REGISTRAZIONE----------------------------------------------------------------
	TA0F1  & Il sistema deve permettere ad un nuovo utente di registrarsi tramite Amazon. All'utente viene chiesto di: 
	\begin{itemize}
		\item verificare che sia possibile premere sul pulsante per la registrazione;
		\item verificare che sia possibile accedere alla pagina di registrazione di Amazon;
		\item inserire tutti i dati per la registrazione;
		\item verificare che a completamento della registrazione si venga reindirizzati alla pagina principale dell'applicativo.
	\end{itemize} 																																								& n.i. \\ 
%lOGIN------------------------------------------------------------------------
	TA0F2  & Il sistema deve permettere all'utente di effettuare il login tramite Amazon. All'utente viene chiesto di:
	 \begin{itemize}
		\item verificare che sia possibile premere sul pulsante per accedere al sistema;
		\item verificare che si venga reindirizzati alla pagina di login di Amazon;
		\item inserire email e password per fare il login;
		\item verificare che si venga reindirizzati alla homepage dell’applicazione dopo il login.
	\end{itemize}  																																								& n.i. \\ 
%CREAZIONE WORKFLOW------------------------------------------------------------			
	TA0F3  & Il sistema deve permettere all'utente autenticato la creazione di un workflow personalizzato. All'utente viene chiesto di:
	\begin{itemize}			
		\item inserire il nome del workflow;			
		\item modificare il messaggio di benvenuto;			
		\item selezionare i blocchi desiderati tra quelli disponibili;			
		\item confermare la configurazione del workflow;			
		\item verificare l'avvenuta creazione del workflow.			
	\end{itemize}  																												      											& n.i. \\                             
	TA0F3  & L'utente autenticato deve poter configurare un blocco tra quelli selezionati. All'utente viene chiesto di:
	\begin{itemize}			
		\item scegliere il tipo di blocco da inserire;			
	\end{itemize}                                           		    			                                                  											& n.i. \\
	TAOF3   & L'applicazione deve visualizzare eventuali errori riguardanti la configurazione del workflow.                            											& n.i. \\
	TA0F4   & L'utente autenticato deve poter modificare i workflow creati precedentemente. All'utente viene chiesto di: 
	\begin{itemize}			
		\item selezionare il workflow da modificare;			
		\item modificare il nome;			
		\item modificare il testo di benvenuto;	
		\item verificare l'avvenuta modifica.			
	\end{itemize}   																												   											& n.i. \\
	TA0F5  & L'utente autenticato deve poter eliminare i workflow creati precedentemente. All'utente viene chiesto di: 
	\begin{itemize}			
		\item selezionare il workflow da eliminare;			
		\item cliccare sul pulsante di eliminazione;			
		\item cliccare sul pulsante per confermare;			
		\item verificare l'avvenuta eliminazione.			
	\end{itemize}     																												   											& n.i. \\
	TA0F6  & L'utente autenticato deve poter ricercare un workflow specifico tra quelli disponibili tramite chiave di ricerca. All'utente viene chiesto di: 
	\begin{itemize}
		\item inserire una keyword nella barra di ricerca;
		\item avviare la ricerca;
		\item verificare che sia possibile ordinare e/o filtrare i risultati.
	\end{itemize}        																											   											& n.i. \\
	TA0F7  & L'utente autenticato deve poter visualizzare i workflow precedentemente creati.                                        											& n.i. \\
%BLOCCO FEED RSS--------------------------------------------------------------											
	TA0F8  & L'utente autenticato deve poter inserire nel workflow un \BFeedRSS. All'utente viene chiesto di:
	 \begin{itemize}											
		\item selezionare il \BFeedRSS;											
		\item inserire l'indirizzo di un \BFeedRSS											
		\item verificare l'avvenuto inserimento del blocco.											
	\end{itemize}    																												   											& n.i. \\
%BLOCCO METEO----------------------------------------------------------------------											
	TA0F9  & L'utente autenticato deve poter inserire nel workflow un \BMeteo. All'utente viene chiesto di: 
	\begin{itemize}											
		\item selezionare il \BMeteo;											
		\item selezionare la scala termometrica desiderata;											
		\item selezionare una certa località tramite geolocalizzazione oppure inserire manualmente il nome della città;											
		\item verificare l'avvenuto inserimento del blocco.											
	\end{itemize}   																												   											& n.i. \\
%BLOCCO INSTAGRAM----------------------------------------------------------------											
	TA0F10   & L'utente autenticato deve poter inserire nel workflow un \BInstagram. All'utente viene chiesto di: 
	\begin{itemize}											
		\item selezionare il \BInstagram;											
		\item inserire le credenziali d'accesso a Instagram;											
		\item verificare l'avvenuto accesso;											
		\item verificare l'avvenuto inserimento del blocco											
	\end{itemize} & n.i. \\                                          											
	TA0F11  & L'utente autenticato tramite Amazon Alexa deve poter ascoltare le notifiche di Instagram.                                       											& n.i. \\
%BLOCCO FACEBOOK-----------------------------------------------------------------										
	TA0F12  & L'utente autenticato deve poter inserire nel workflow un \BFacebook. All'utente viene chiesto di: \begin{itemize}										
		\item selezionare il \BFacebook;										
		\item inserire le credenziali d'accesso a Facebook;										
		\item verificare l'avvenuto accesso;										
		\item verificare l'avvenuto inserimento del blocco.										
	\end{itemize} & n.i. \\										
	TA0F13   & L'utente autenticato tramite Alexa deve poter ascoltare le notifiche di Facebook non ancora lette.                        										& n.i. \\
	TA1F14   & L'utente autenticato tramite Alexa deve poter ascoltare gli ultimi post di Facebook della bacheca.                      										& n.i. \\
%BLOCCO LINKEDIN-----------------------------------------------------------------										
	TA0F15   & L'utente autenticato deve poter inserire nel workflow un \BLinkedIn. All'utente viene chiesto di: 
	\begin{itemize}										
		\item selezionare il \BLinkedIn;										
		\item inserire le credenziali d'accesso di LinkedIn;										
		\item verificare l'avvenuto accesso;										
		\item verificare l'avvenuto inserimento del blocco.										
	\end{itemize} 																													    										& n.i. \\										
	TA1F16   & L'utente autenticato tramite Amazon Alexa deve poter ascoltare le notifiche di LinkedIn                                         										& n.i. \\
	TA0F16   & L'utente autenticato tramite Amazon Alexa deve poter ascoltare gli ultimi messaggi di LinkedIn                                  											& n.i. \\
%BLOCCO MESSENGER--------------------------------------------------------------- 										
	TA0F17   & L'utente autenticato deve poter inserire nel workflow un \BMessenger. All'utente viene chiesto di: 
	\begin{itemize}										
		\item selezionare il \BMessenger;										
		\item inserire le credenziali d'accesso a Facebook Messenger;										
		\item verificare l'avvenuto accesso;										
		\item verificare l'avvenuto inserimento del blocco.										
	\end{itemize} 																	                                                    										& n.i. \\
	TA0F18   & L'utente autenticato deve poter inserire le credenziali d'accesso a Facebook Messenger.                          										& n.i. \\
	TA0F19   & L'utente autenticato tramite Amazon Alexa deve poter ascoltare gli ultimi messaggi di Facebook Messenger.                    										& n.i. \\
%BLOCCO SVEGLIA------------------------------------------------------------------										
	TA0F20   & L'utente autenticato deve poter inserire nel workflow un \BSveglia. All'utente viene chiesto di: \begin{itemize}										
		\item selezionare il \BSveglia;										
		\item impostare l'orario di una sveglia;										
		\item scegliere la suoneria per la sveglia;										
		\item verificare l'avvenuto inserimento.										
	\end{itemize} 																																								& n.i. \\  	
	TA0F21   & L'utente autenticato deve poter modificare il workflow un \BSveglia. All'utente viene chiesto di: \begin{itemize}										
		\item selezionare il \BSveglia;										
		\item modificare l'orario di una sveglia;										
		\item modificare la suoneria per la sveglia;										
		\item verificare l'avvenuta modifica.										
	\end{itemize}    																																							& n.i. \\
	%TA0F23   & L'utente autenticato deve poter fermare la sveglia mentre questa sta suonando                                            										& n.i. \\
%BLOCCO MAIL-------------------------------------------------------------------										
	TA0F22   & L'utente autenticato deve poter inserire nel workflow un \BMail. All'utente viene chiesto di: \begin{itemize}										
		\item selezionare il \BMail;										
		\item inserire le credenziali d'accesso di Gmail;										
		\item verificare l'avvenuto accesso;										
		\item verificare l'avvenuto inserimento del blocco.										
	\end{itemize} 																																								& n.i. \\        
	TA0F23   & L'utente tramite Amazon Alexa deve poter leggere le ultime mail non ancora lette.                                              										& n.i. \\
	TA0F24   & Amazon Alexa deve riferire all'utente il numero di mail non ancora lette.                                                										& n.i. \\
%BLOCCO TELEGRAM---------------------------------------------------------------										
	TA0F25   & L'utente autenticato deve poter inserire nel workflow un \BTelegram. All'utente viene chiesto di: \begin{itemize}										
		\item selezionare il \BTelegram;										
		\item inserire le credenziali d'accesso a Telegram										
		\item verificare l'avvenuto accesso;										
		\item verificare l'avvenuto inserimento del blocco.										
	\end{itemize} 																																								& n.i. \\       
	TA0F26   & L'utente autenticato tramite Amazon Alexa deve poter leggere gli ultimi messaggi di una persona o di un gruppo.                     										& n.i. \\
	TA0F27   & L'utente autenticato tramite Amazon Alexa potrà riprodurre gli ultimi messaggi vocali di una persona o di un gruppo.                										& n.i. \\
%BLOCCO SLACK------------------------------------------------------------------------										
	TA0F28   & L'utente autenticato deve poter inserire nel workflow un \BSlack\textsubscript{G}. All'utente viene chiesto di: \begin{itemize}										
		\item selezionare il \BSlack;										
		\item inserire le credenziali d'accesso a Slack										
		\item verificare l'avvenuto accesso;										
		\item verificare l'avvenuto inserimento del blocco.										
	\end{itemize} & n.i. \\    	                                          										
	TA0F29   & L'utente autenticato tramite Amazon Alexa deve poter leggere gli ultimi messaggi di un certo canale.                             										& n.i. \\
	TA0F30   & L'utente autenticato deve poter selezionare il canale di cui leggere i messaggi.                                          										& n.i. \\
%BLOCCO CALENDARIO------------------------------------------------------------------										
	TA0F31   & L'utente autenticato deve poter inserire nel workflow un \BCalendario. All'utente viene chiesto di: \begin{itemize}										
		\item selezionare il \BCalendario;										
		\item inserire le credenziali d'accesso a Google Calendar;										
		\item verificare l'avvenuto accesso;										
		\item verificare l'avvenuto inserimento del blocco.										
	\end{itemize} 																																								& n.i. \\   
	TA0F32   & L'utente autenticato tramite Amazon Alexa deve poter leggere gli eventi giornalieri.                                             										& n.i. \\
	TA0F33   & L'utente autenticato deve poter inserire un evento all'interno di Google Calendar.                                        										& n.i. \\
%BLOCCO YOUTUBE------------------------------------------------------------------										
	TA0F34   & L'utente autenticato deve poter inserire nel workflow un \BYouTube. All'utente viene chiesto di: \begin{itemize}										
		\item selezionare il \BYouTube;										
		\item inserire le credenziali d'accesso a YouTube;										
		\item verificare l'avvenuto accesso;										
		\item verificare l'avvenuto inserimento del blocco.										
	\end{itemize} 																																								& n.i. \\ 
	TA0F35   & L'utente autenticato tramite Amazon Alexa deve poter riprodurre gli ultimi video di un canale YouTube.                           										& n.i. \\
	TA0F36   & L'utente autenticato deve poter inserire modificare il link del canale YouTube.                                          										& n.i. \\
%BLOCCO YOUTUBE MUSIC-----------------------------------------------------------										
	TA0F37   & L'utente autenticato deve poter inserire nel workflow un \BYouTubeMusic. All'utente viene chiesto di: \begin{itemize}										
		\item selezionare il \BYouTubeMusic;										
		\item inserire le credenziali d'accesso a YouTube Music;										
		\item verificare l'avvenuto accesso;										
		\item verificare l'avvenuto inserimento del blocco.										
	\end{itemize} 																																								& n.i. \\ 
	TA0F38   & L'utente autenticato tramite Amazon Alexa deve poter riprodurre le canzoni di una playlist YouTube Music.                       										& n.i. \\
	TA0F39   & L'utente autenticato deve poter modificare il link della playlist, del canale o della canzone YouTube Music.             										& n.i. \\
%BLOCCO RADIO-----------------------------------------------------------										
	TA0F40   & L'utente autenticato deve poter inserire nel workflow un \BRadio. All'utente viene chiesto di: \begin{itemize}										
		\item selezionare il \BRadio;										
		\item inserire l'indirizzo del Feed RSS di un canale radio;										
		\item verificare l'avvenuto inserimento del blocco.										
	\end{itemize} 																																								& n.i. \\ 
	TA0F41  & L'utente autenticato deve poter scegliere il canale radio preferito tra quelli disponibili.                               										& n.i. \\
%BLOCCO PROGRAMMAZIONE TV-----------------------------------------------------------										
	TA0F42   & L'utente autenticato deve poter inserire nel workflow un \BTV. All'utente viene chiesto di: \begin{itemize}										
		\item selezionare il \BTV;										
		\item inserire l'indirizzo del Feed RSS di un canale televisivo;										
		\item verificare l'avvenuto inserimento del blocco.										
	\end{itemize} 																																								& n.i. \\ 	
	TA0F43   & L'utente autenticato deve poter scegliere il canale televisivo preferito tra quelli disponibili.                          										& n.i. \\
%BLOCCO SPOTIFY-----------------------------------------------------------    										
	TA0F44   & L'utente autenticato deve poter inserire nel workflow un \BSpotify. All'utente viene chiesto di: \begin{itemize}										
		\item selezionare il \BSpotify;										
		\item inserire le credenziali d'accesso a Spotify;										
		\item verificare l'avvenuto accesso;										
		\item verificare l'avvenuto inserimento del blocco.										
	\end{itemize} 																																								& n.i. \\  
	TA0F45   & L'utente autenticato tramite Amazon Alexa deve poter riprodurre una canzone.                                               										& n.i. \\
	TA0F46   & L'utente autenticato tramite Amazon Alexa deve poter riprodurre una playlist.                                             										& n.i. \\
%BLOCCO CINEMA-----------------------------------------------------------      										
	TA0F47   & L'utente autenticato deve poter inserire nel workflow un \BCinema. All'utente viene chiesto di: \begin{itemize}										
		\item selezionare il \BCinema;										
		\item inserire testualmente il nome di un cinema o selezionare quello più vicino tra quelli disponibili tramite geolocalizzazione										e
		\item verificare l'avvenuto inserimento del blocco.										
	\end{itemize} 																																								& n.i. \\  
	TA0F48   & L'utente autenticato deve poter modificare il nome del cinema più vicino.                                                 										& n.i. \\
	TA0F49   & L'utente autenticato tramite Amazon Alexa deve sapere la programmazione odierna di un certo cinema.                              										& n.i. \\
	TA0F50   & L'utente autenticato tramite Amazon Alexa deve sapere la programmazione di una determinata ora di un certo cinema.                              										& n.i. \\
%BLOCCO TRASPORTI-----------------------------------------------------------   										
	TA0F51   & L'utente autenticato deve poter inserire nel workflow un \BTrasporti. All'utente viene chiesto di: \begin{itemize}										
		\item selezionare il \BTrasporti;										
		\item selezionare la località di partenza;										
		\item selezionare la località di arrivo;																				
		\item verificare l'avvenuto inserimento del blocco.										
	\end{itemize} 																																								& n.i. \\
	TA0F52   & L'utente autenticato deve poter modificare la località di partenza.                                                       										& n.i. \\
	TA0F53   & L'utente autenticato deve poter modificare la località di arrivo.                                                        										& n.i. \\
	TA0F54   & L'utente autenticato tramite Amazon Alexa sa qual è il prossimo treno disponibile.                                            										& n.i. \\
%BLOCCO LISTA-----------------------------------------------------------   										
	TA0F55   & L'utente autenticato deve poter inserire nel workflow un \BLista. All'utente viene chiesto di: 										
	\begin{itemize}										
		\item selezionare il \BLista;										
		\item aggiungere elementi alla lista;										
		\item verificare l'avvenuta creazione della lista.										
		\item verificare l'avvenuto inserimento del blocco.										
	\end{itemize}   																																						                                         										& n.i. \\
	TA0F56   & L'utente autenticato tramite Amazon Alexa deve sapere quali elementi ci sono nella lista.                                 										& n.i. \\
%BLOCCO SICUREZZA------------------------------------------------------ 										
	TA0F57   & L'utente autenticato deve poter inserire nel workflow un \BSicurezza. All'utente viene chiesto di: 										
	\begin{itemize}										
		\item selezionare il \BSicurezza;										
		\item aggiungere un PIN;										
		\item verificare l'avvenuta creazione del blocco sicurezza.						
	\end{itemize}   																																							& n.i. \\
%BLOCCO KINDLE
	TA0F58 & L'utente autenticato deve poter inserire nel workflow un \BKindle. All'utente viene chiesto di:
	\begin{itemize}										
		\item selezionare il \BKindle;										
		\item aggiunge un link o inserire un file PDF/EPUB;										
		\item verificare l'avvenuta creazione del \BKindle.						
	\end{itemize}   										& n.i\\
%LOGOUT
	TA0F59   & L'utente autenticato deve poter effettuare il logout. All'utente viene chiesto di: 										
	\begin{itemize}										
		\item cliccare sul pulsante di logout;						
		\item verificare l'avvenuto logout.						
	\end{itemize}   												                                                                     										& n.i. \\

\end{longtable}

\end{center}

