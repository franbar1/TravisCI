\section{Resoconto}

    \subsection{Scelta e analisi dei tipi di blocco da implementare}
    Il progetto proposto nel capitolato C4 prevede la creazione di workflow per Alexa. Questi workflow conterranno dei blocchi specifici che eseguiranno determinate operazioni. Si è scelto di rendere disponibile all'utente i seguenti blocchi:
    \begin{itemize}
        \item Testo;
        \item Feed RSS;
        \item Filtro;
        \item Meteo;
        \item Instagram;
        \item Facebook;
        \item Messenger;
        \item LinkedIn;
        \item Sveglia;
        \item Slack;
        \item Telegram;
        \item Mail;
        \item Calendario;
        \item YouTube;
        \item YouTube Music;
        \item Radio;
        \item Programmazione TV;
        \item Spotify;
        \item Cinema;
        \item Trasporto;
        \item Lista;
        \item Sicurezza;
        \item Kindle.
    \end{itemize}

    \subsection{Riorganizzazione repository GitLab}
    Per gestire al meglio i vari documenti prodotti e in vista delle successive consegne per le RA, il gruppo ha deciso di riorganizzare la repository, con una suddivisione più chiara e uniforme. Ci saranno 4 cartelle corrispondenti alle RA, ognuna delle quali con i relativi prodotti.