\section{Qualità di prodotto}
Per garantire una buona qualità di prodotto, il gruppo ha individuato dallo standard 
\glo{ISO/IEC 9126} le caratteristiche di qualità necessarie affinché il prodotto finale sia di buona qualità. 
Inoltre il gruppo ha assegnato ad ogni sotto-caratteristica un grado di importanza (basso, medio alto), 
al fine di valutare al meglio il prodotto finale. Tali valori sono riportati in tabella.

\begin{center}
  \renewcommand{\arraystretch}{2}
   \begin{longtable}{m{3cm} m{6cm} m{3cm}}

       \rowcolor[HTML]{232f3e}

 \color[HTML]{FFFFFF}\textbf{Caratteristica} & \color[HTML]{FFFFFF}\textbf{Attributi} & \color[HTML]{FFFFFF}\textbf{Importanza} \

\endhead
     \rowcolor{white}\multicolumn{3}{c}
 { Continua nella pagina successiva} \\
 \endfoot

 \caption [Caratteristiche e valori per la qualità]{Caratteristiche e valori per la qualità}
   \label{tabella:qualProd}
 \endlastfoot

Funzionalità & Completezza & Alta\\
 & Accuratezza  & Alta\\
  & Interoperabilità & Alta\\
  & Sicurezza & Media\\
\rowcolor{tableRow}Affidabilità & Tolleranza ai guasti & Alta\\
Usabilità & Comprensibilità & Alta\\
 & Apprendibilità & Media\\
\rowcolor{tableRow}Efficienza & Comportamento rispetto al tempo & Alta\\
 Manutenibilità & Modificabilità & Media\\
               & Stabilità & Alta\\
                & Testabilità & Media\\


   \end{longtable}

\end{center}

    \subsection{Qualità dei documenti}
    Tutti i documenti prodotti del gruppo Hexadec devono essere leggibili, comprensibili e corretti dal punto di vista ortografico, sintattico, logico e semantico. Verranno utilizzate le seguenti metriche %definite nelle Norme di progetto in sezione
    \begin{itemize}
        \item Indice di Gulpease;
        \item Errori ortografici.
    \end{itemize}
    
    \subsection{Qualità del software}
        \subsubsection{Funzionalità}
        Rappresenta la capacità del software di soddisfare tutte le funzionalità che sono state individuate attraverso l’ \docNameVersionAdR{}.
        \paragraph{Obiettivi di Qualità}
        \begin{itemize}
            \item \textbf{Completezza}: le funzionalità sviluppate devono coprire tutte quelle richieste dal committente e quelle individuate dal gruppo;
            \item \textbf{Accuratezza}: il prodotto deve produrre i risultati attesi e con un buon grado di precisione;
            \item \textbf{Interoperabilità}: la capacità del prodotto di interagire con gli altri sistemi definiti è essenziale per la riuscita del progetto.
        \end{itemize}
        Verranno utilizzate le seguenti metriche:
        
        \begin{center}
            \renewcommand{\arraystretch}{2.2}
            \rowcolors{3}{tableRow}{}
            
            \begin{longtable}{c c c c}
              
              \rowcolor[HTML]{232f3e}
            
              \rowcolors{1}{tableRow}{}
              \color[HTML]{FFFFFF} \textbf{Metriche} & \color[HTML]{FFFFFF} \centering\textbf{Range di accettazione} & \color[HTML]{FFFFFF} \centering\textbf{Range di ottimalità} & \color[HTML]{FFFFFF} \centering\textbf{Unità di misura} 
            \endhead
            \rowcolor{white}\multicolumn{1}{c}
               { Continua nella pagina successiva} \\
               \endfoot
               \caption [Range delle metriche riguardanti le funzionalità del prodotto]{Range delle metriche riguardanti le funzionalità del prodotto}
              \label{tabella:reqP1}
               \endlastfoot
               %sv
               Copertura dei requisiti obbligatori & 100 & 100 & Percentuale \\
               Copertura dei requisiti desiderabili & [60,100] & [80,100] & Percentuale  \\

            \end{longtable}

        \end{center}

        \subsubsection{Affidabilità}
        Rappresenta la capacità del prodotto software di svolgere correttamente le sue funzioni durante il suo utilizzo, anche in caso in cui si presentino situazioni anomale.
        \paragraph{Obiettivi di Qualità}
        \begin{itemize}
            \item \textbf{Tolleranza ai guasti}:  è necessario gestire al meglio eventuali malfunzionamenti o utilizzi scorretti dell'applicativo.
        \end{itemize}
        Verranno utilizzate le seguenti metriche:
        \begin{center}
            \renewcommand{\arraystretch}{2.2}
            \rowcolors{3}{tableRow}{}
            
            \begin{longtable}{c c c c }
              
              \rowcolor[HTML]{232f3e}
            
              \rowcolors{1}{tableRow}{}
              \color[HTML]{FFFFFF} \textbf{Metriche} & \color[HTML]{FFFFFF} \centering\textbf{Obiettivo} & \color[HTML]{FFFFFF} \centering\textbf{Range di ottimalità} & \color[HTML]{FFFFFF} \centering\textbf{Unità di misura} 
            \endhead
            \rowcolor{white}\multicolumn{1}{c}
               { Continua nella pagina successiva} \\
               \endfoot
               \caption [Range delle metriche riguardanti l'affidabilità del prodotto]{Range delle metriche riguardanti l'affidabilità del prodotto}
              \label{tabella:reqP1}
               \endlastfoot
               %sv
               Tolleranza ai guasti & [0,10] & 0 & Percentuale \\

            \end{longtable}

        \end{center}
       
        \subsubsection{Usabilità}
        Rappresenta la capacità del prodotto di essere comprensibile e facile da utilizzare 
        da ogni utente.
        \paragraph{Obiettivi di Qualità}
        \begin{itemize}
            \item \textbf{Comprensibilità}: il prodotto deve essere di facile comprensione in aderenza all'utilizzo che un utente intende farne;
            \item \textbf{Apprendibilità}: dovrà essere disponibile un help on-line che aiuti l'utente a comprendere le funzionalità dell’applicativo.
        \end{itemize}
        Verranno utilizzate le seguenti metriche:
        \begin{center}
            \renewcommand{\arraystretch}{2.2}
            \rowcolors{3}{tableRow}{}
            
            \begin{longtable}{c c c c }
              
              \rowcolor[HTML]{232f3e}
            
              \rowcolors{1}{tableRow}{}
              \color[HTML]{FFFFFF} \textbf{Metriche} & \color[HTML]{FFFFFF} \centering\textbf{Obiettivo} & \color[HTML]{FFFFFF} \centering\textbf{Range di ottimalità} & \color[HTML]{FFFFFF} \centering\textbf{Unità di misura} 
            \endhead
            \rowcolor{white}\multicolumn{1}{c}
               { Continua nella pagina successiva} \\
               \endfoot
               \caption [Range delle metriche riguardanti l'usabilità del prodotto]{Range delle metriche riguardanti l'usabilità del prodotto}
              \label{tabella:reqP1}
               \endlastfoot
               %sv
               Comprensibilità delle funzionalità offerte & [70-100] & [90,100] & Percentuale \\

            \end{longtable}

        \end{center}
       
        \subsubsection{Efficienza}
        Rappresenta la capacità di eseguire le funzionalità offerte dal prodotto nel minor tempo possibile e utilizzando al tempo stesso il minor numero di risorse disponibili.
        \paragraph{Obiettivi di Qualità}
        \begin{itemize}
            \item \textbf{Comportamento rispetto al tempo}: il prodotto dovrà necessariamente avere un tempo di risposta minore o uguale a quello richiesto dal committente.
        \end{itemize}
        Verranno utilizzate le seguenti metriche:
        \begin{center}
            \renewcommand{\arraystretch}{2.2}
            \rowcolors{1}{tableRow}{}
            
            \begin{longtable}{c c c c }
              
              \rowcolor[HTML]{232f3e}
            
              \rowcolors{1}{tableRow}{}
              \color[HTML]{FFFFFF} \textbf{Metriche} & \color[HTML]{FFFFFF} \centering\textbf{Obiettivo} & \color[HTML]{FFFFFF} \centering\textbf{Range di ottimalità} & \color[HTML]{FFFFFF} \centering\textbf{Unità di misura} 
            \endhead
            \rowcolor{white}\multicolumn{1}{c}
               { Continua nella pagina successiva} \\
               \endfoot
               \caption [Range delle metriche riguardanti l'efficienza del prodotto]{Range delle metriche riguardanti l'efficienza del prodotto}
              \label{tabella:reqP1}
               \endlastfoot

               Tempo di risposta & $\leq$ 90 & [0,15] & Secondi \\

            \end{longtable}

        \end{center}
       
        \subsubsection{Manutenibilità}
        Rappresenta il livello di impegno richiesto per modificare il prodotto, attraverso correzioni o miglioramenti nei requisiti e nelle specifiche funzionali.
        \paragraph{Obiettivi di Qualità}
        \begin{itemize}
            \item \textbf{Modificabilità}: il prodotto originale deve permettere eventuali modifiche in alcune sue parti;
            \item \textbf{Stabilità}: si deve garantire che non insorgano errori in seguito a modifiche effettuate sul software;
            \item \textbf{Testabilità}: il software deve poter essere facilmente testato per validare le modifiche effettuate.
        \end{itemize}
        Verranno utilizzate le seguenti metriche:
        \begin{center}
            \renewcommand{\arraystretch}{2.2}
            \rowcolors{1}{tableRow}{}
            
            \begin{longtable}{c c c c }
              
              \rowcolor[HTML]{232f3e}
            
              \rowcolors{1}{tableRow}{}
              \color[HTML]{FFFFFF} \textbf{Metriche} & \color[HTML]{FFFFFF} \centering\textbf{Obiettivo} & \color[HTML]{FFFFFF} \centering\textbf{Range di ottimalità} & \color[HTML]{FFFFFF} \centering\textbf{Unità di misura} 
            \endhead
            \rowcolor{white}\multicolumn{1}{c}
               { Continua nella pagina successiva} \\
               \endfoot
               \caption [Range delle metriche riguardanti la manutenibilità del prodotto]{Range delle metriche riguardanti la manutenibilità del prodotto}
              \label{tabella:reqP1}
               \endlastfoot

               Capacità analisi failure &  [60,100] & [80,100] & Percentuale \\
				 Impatto delle modifiche & [0,20] & [0,10] & Percentuale \\
            \end{longtable}

        \end{center}
       \pagebreak
