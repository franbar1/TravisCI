\section{\quad$P\quad$}
\subsection{Part of Speech-tagger}
\index{Part of Speech-tagger}
Un software che legge il testo in alcune lingue e assegna parti del discorso a ogni parola con il suo significato grammaticale.

\subsection{PDCA}
\index{PDCA}
Il ciclo di Deming (PDCA : Plan, do, Check, Act) è un metodo di gestione iterativo in quattro fasi utilizzato per il controllo e il miglioramento continuo dei processi e dei prodotti.


\subsection{Peer-to-peer}
\index{Peer-to-peer}
Una rete P2P è un’infrastruttura in cui l’accesso ai dati avviene senza utilizzare server dedicati. Ogni dispositivo connesso, detto peer, può ricoprire sia il ruolo di client che il ruolo di server.

\subsection{Plug-in}
\index{Plug-in}
Un Plug-In è un modulo aggiuntivo di un programma, utilizzato per aumentarne le funzioni.

\subsection{Processo}
\index{Processo}
Insieme di attività correlate e coese che trasformano ingressi (bisogni) in uscite (prodotti) secondo regole date, consumando risorse nel farlo.

\subsection{Product Baseline}
\index{Product Baseline}
Baseline del progetto in cui il cui progetto è realizzato e funziona correttamente ma non è ancora pronto al rilascio.

\subsection{Proof of Concept}
\index{Proof of Concept}
È una realizzazione incompleta o abbozzata di un determinato progetto, allo scopo di provarne la fattibilità. Nell'ambito informatico si riferisce alla dimostrazione pratica dei funzionamenti di base di un applicativo software.

\subsection{Python}
\index{Python}
Linguaggio di programmazione ad alto livello, orientato agli oggetti, usato per sviluppare applicazioni distribuite, scripting, computazione numerica e system testing.
